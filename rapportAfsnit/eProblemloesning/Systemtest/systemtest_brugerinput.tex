\section{Systemtest med bruger-input}

\subsection{Beskrivelse}
For at teste det samlede system med bruger-input påsættes de positive og negative elektroder tibia, \autoref{fig:laarmuskler}, og referenceelektroden påsættes anklen, \autoref{fig:reference}, ud fra SENIAMs anvisninger \citep{seniam2016}, som også benyttes til elektrodeplacering i pilotforsøget i \autoref{sec:pilotforsoeg}. Huden præpareres inden dette for at fjerne hår og døde hudceller, der vil kunne påvirke elektrodernes evne til at opsamle data. 
De to accelerometre, der, som i pilotforsøget, påsættes breadboards for at stabilisere deres placere mest muligt, placeres hhv. parallelt med femur og tibia, som illustreret på \autoref{fig:accelerometervinkel}.

Systemet testes over 10 sekunders målinger, hvor forsøgspersonen udfører squats. Der skal hertil laves to målinger, hvor forsøgspersonen holder sig inden for en vinkel over knæet på $90-180^{\circ}$, to målinger, hvor forsøgspersonens vinkel over knæet er $<90^{\circ}$, og to målinger, hvor forsøgspersonens vinkel over knæet er $>180^{\circ}$. På denne måde testes det om det samlede system fungerer, da en overskridelse af $90-180^{\circ}$ vil betyde, at data ikke vil gå videre ind i EMG-algoritmen. Yderligere vil testen vise, om EMG-algoritmen også fungerer med EMG-signaler, når hele systemet er koblet sammen. 

\subsection{Udførsel af test}


\subsection{Konklusion}
