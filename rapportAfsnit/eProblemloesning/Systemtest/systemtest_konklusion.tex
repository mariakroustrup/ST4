\section{Konklusion af systemtest}
Da det ikke er alle krav i \autoref{sec:overordnet_krav}, der er mulighed for at teste og dokumenterer ud fra de forsøg , der er foretaget, vurderes disse på baggrund af implementering, test og videreudvikling. 

Det samlede system var batteridrevet under udførelsen af begge forsøg, hvorfor dette krav er overholdt. Derudover var det muligt at overføre de beregnede data i mikrokontrollen trådløst til en computer, hvor de efterfølgende blev visualiseret i MATLAB i realtime. På baggrund af dette vurderes det, at det er muligt at sende data trådløst fra en computer videre til et exoskelet, bygget i LEGO mindstorm NXT. Det vurderes yderligere, at systemet er sikkert i forhold til brugersikkerheden, da systemet under forsøget var batteridrevet, og derfor ikke var tilkoblet elnettet, hvormed dette mindskede muligheden for lækstrøm. 
På nuværende tidspunkt kan systemet opfattes som ikke værende optimalt for brugeren, da der ikke er udarbejdet en prototype. Herved er kravet om exoskelet ikke skal være til gene for brugeren ikke er opfyldt. Ved udvikling af en prototype vil der skulle tages højde dette. \fxnote{eventuelt en ref til perspektivering - måske et billede af hvordan vi tænker det vil kunne komme til at se ud?}. 
På baggrund af målinger foretaget i \autoref{test_spaendingsforsyning} blev det påvist, at spændingsregulatoren indikerer, når systemet ikke leverer den optimale strøm til systemet, hvorved dette krav er opfyldt. 

Ud fra de to forsøg med henholdsvis kendt input og bruger-input vurderes det, at systemet fungerer under kontrollerede forhold samt ved anvendelse af en bruger. Systemet kan opsamle signaler fra rectus femoris samt beregne vinklen af knæleddet, hvilket blev påvist i forsøget med brugerinput, hvor resultaterne fremgår af \autoref{fig:test_brugerinput}. Baseret på disse resultater vurderes det yderligere, at systemet kan følge kroppens naturlige bevægelse under udførsel af en squat-øvelse. 

På baggrund af dette vurderes det at systemet overholder de overordnede krav opstillet i \autoref{sec:overordnet_krav}. 

