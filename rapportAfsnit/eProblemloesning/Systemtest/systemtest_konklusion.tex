\section{Konklusion af systemtest}
Da det ikke er muligt at teste og dokumentere alle krav i \autoref{sec:overordnet_krav}, vurderes disse på baggrund af implementering, test og videreudvikling. 

\noindent
Ud fra de to forsøg med henholdsvis kendt og bruger-input vurderes det, at systemet fungerer under kontrollerede forhold samt med input fra en bruger. Systemet kan opsamle signaler fra rectus femoris samt beregne vinklen af knæleddet, hvilket blev påvist i forsøget med bruger-input, hvor resultaterne fremgår af \autoref{fig:test_brugerinput}.
Baseret på disse resultater vurderes det yderligere, at systemet kan følge kroppens naturlige bevægelse under udførsel af en squat-øvelse, så det er muligt at benytte dette til en prototype af et exoskelet. 

På nuværende tidspunkt er der udviklet et kontrolsystem, der anvender overfladeelektroder og accelerometre.
Ved påsætning af disse er der taget hensyn til brugeren, på trods af at påsætningen af accelerometrene under test af systemet ikke er hensigtsmæssig. Ved kombination af det udviklede kontrolsystem og et exoskelet fikseres accelerometrene på exoskelellet for således at gøre systemet til mindst mulig gene for brugeren. 

Det samlede system var batteridrevet under udførelsen af begge forsøg, hvorfor dette krav er overholdt. Derudover var det muligt at overføre data fra mikrokontrollen trådløst til en computer, hvor de efterfølgende blev visualiseret i MATLAB i realtid. På baggrund af dette vurderes det, at det er muligt at sende data trådløst fra en computer videre til et exoskelet. Det vurderes yderligere, at systemet er brugersikkert, da systemet under forsøget blandt andet var batteridrevet, og derfor ikke var tilkoblet elnettet, hvormed dette mindskede muligheden for lækstrøm, hvilket er beskrevet i \autoref{sec:brugersikkerhed}.

Derudover blev det, på baggrund af målinger foretaget i \autoref{test_spaendingsforsyning}, påvist, at spændingsregulatoren indikerer, når systemet ikke leverer den optimale strøm til systemet, hvorved dette krav er opfyldt. 

Da systemet er tiltænkt ALS-patienter vurderes det, at forsinkelsen på $832~\mu s$ over det samlede system uden trådløs kommunikation ikke har en betydning i forhold til at følge kroppens naturlige bevægelse. Dermed får dette ikke en betydning for systemet, hvorfor denne forsinkelse accepteres. 

På baggrund af dette vurderes det, at systemet overholder de overordnede krav opstillet i \autoref{sec:overordnet_krav}. 

\vspace{3mm}
\textbf{Opsummering af krav:}
\begin{itemize}
\item[\text{\sffamily \checkmark}] Systemet skal registrere muskelaktivitet fra rectus femoris 
\item[\text{\sffamily \checkmark}] Systemet skal måle vinklen over knæet
\item[\text{\sffamily \checkmark}] Systemet skal reagere på kroppens bevægelse under en squat-øvelse, således det vil kunne benyttes til en prototype af et exoskelet
\item[\text{\sffamily \checkmark}] Systemet skal være sikkert og ikke til gene for brugeren 
\item[\text{\sffamily \checkmark}] Systemet skal kunne overføre data trådløst til en computer
\item[\text{\sffamily \checkmark}] Systemet skal være batteridrevet
\item[\text{\sffamily \checkmark}] Systemet skal have en maksimal forsinkelse af hele systemet på $100~ms$
%\item[\text{\sffamily \checkmark}] Systemet skal kunne ende ud i en prototype af et exoskelet
\end{itemize}
