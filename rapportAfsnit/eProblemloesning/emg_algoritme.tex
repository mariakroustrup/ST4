\subsection{EMG-algoritme}\label{sec:krav_emg_algo}
EMG-algoritmen har til formål at styre fleksion samt ekstension af en prototype, og dermed knæleddet udfra rectus femoris' muskelaktivitet. Hertil skal knæet fleksere ved en stigende muskelaktivitet og ekstendere, når muskelaktiviteten er faldende.
Dette bestemmes ved, at EMG-algoritmen skal finde hældningen af EMG-signalet mellem samples og derefter udsende et signal alt efter, om hældningen er aftagende eller stigende. Dette signal skal indikere ændringen i outputsignalet. Ved en stigning af muskelaktiviteten skal outputtet visualisere +10 og ved et fald i muskelaktiviteten visualisere -10.
For at undgå, at prototypen eksentenderer eller flekserer udenfor det definerede område for squat-øvelsen, beskrevet i \autoref{sec:knaeled_squat} ønskes det, at EMG-algoritmen kun virker indenfor en vinkel, hvor knæet befinder sig i intervallet $90-180^{\circ}$. Ved overskridelse af dette interval skal EMG-algoritmen vise et output på 0.


\vspace{3mm}
\textbf{Krav:}
\begin{itemize}
\item Skal kunne detektere om muskelaktiviteten er faldende eller stigende mellem to samples med 0,01 sekunders mellemrum
\begin{itemize}
\item Ved stigende muskelaktivitet skal dette indikeres som et outputsignal på $+10$
\item Ved faldende muskelaktivitet skal dette indikeres som et outputsignal på $-10$
\end{itemize}
\item Skal kunne indikere, hvis vinklen befinder sig undenfor intervallet $90-180^{\circ}$
\begin{itemize}
\item Dette skal indikeres ved, at outputsignalet går i $0$
\end{itemize}
\end{itemize}


