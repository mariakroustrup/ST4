\subsection{EMG-algoritme}
EMG-algoritmen har til formål at få prototypen, og dermed knæleddet, til at fleksere, når muskelaktiviteten fra rectus femoris er stigende, og at få prototypen til at ekstendere, når muskelaktiviteten er faldende. 

For at opfylde dette formål skal hældningen af grafen findes. Dette kan gøres ved differentiering, hvorved det vil være muligt at finde tangenthældningen i ét punkt ved differentialkvotienten. Det vælges at implementere en mere simpel metode til at tilnærmelsesvist at finde hældningen ved \autoref{eq:haeldning}.

\begin{equation}
f'(x)\approx\dfrac{\Delta y(x)}{\Delta x}
\label{eq:haeldning}
\end{equation}

\noindent
I \autoref{eq:haeldning} er $\Delta x$ tiden mellem to samples, og $\Delta y(x)$ den målte spænding fra rectus femoris til tiden $x$. 

Hvis $f'(x)>1$ kommer der et output, som signalerer til prototypen, at knæleddet skal fleksere. Hvis derimod $f'(x)<1$ kommer der et andet output, som signalerer til prototypen, at knæleddet skal ekstendere.