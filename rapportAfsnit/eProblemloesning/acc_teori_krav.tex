\subsection{Accelerometer}
Et accelerometer er en elektromekanisk enhed, som både kan måle statisk og dynamisk accerleration. Den statiske acceleration kan være tyngdekraften, hvor det er muligt at bestemme orienteringen af accelerometeret i forhold til jorden. De dynamiske kræfter såsom bevægelse, stød og vibrationer, gør det muligt at analysere accelerometeres bevægelse samt hastighed. \fxnote{Der skal skrives mere omkring hvorfor den skal forsynes med en spænding og hvorfor det skal kunne måle 1g, hvilket frekvensområde ligger det indenfor?}
\vspace{3mm}

\textbf{Krav:}
\begin{itemize}
\item Skal måle på minimum én akse
\item Skal forsynes med en spænding på $3,3~V$
\item Skal måle accelerationer i $\pm1~g$
%\item Skal give output i form af spænding (analog)
\end{itemize}
