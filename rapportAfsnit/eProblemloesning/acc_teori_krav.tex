\subsection{Opsamling af accelerometer-signaler}
Et accelerometer er en elektromekanisk enhed, som både kan måle statisk og dynamisk accereleration. Den statiske acceleration er på $1~g$, hvilket svarer til tyngdekraften. Alt efter hvilken retning accelerometeret holdes i, ændres aksen, hvor der måles $1~g$-påvirkning i.
Ud fra dette er det muligt at bestemme orienteringen af accelerometeret i forhold til jorden. De dynamiske kræfter såsom bevægelse, stød og vibrationer, gør det muligt at analysere accelerometeres bevægelse samt hastighed. Ved bevægelse udsættes accelerometeret både for dynamisk og statisk acceleration. Studier har vist, at den højest mulige acceleration ved bevægelse af en arm går fra 0,5 til 2,0 g-påvirkning. Herved forventes dette ligeledes for et ben under en squat-øvelse \citep{bernmarka2002}. I dette projekt måles vinkelen af knæet under en squat-øvelse, derfor vil det være mest hensigtsmæssigt at placere accelerometeret, så det måler i enten X- eller Y-asken. På baggrund af dette vælges Y-aksen.

\vspace{3mm}

\textbf{Krav:}
\begin{itemize}
\item Skal måle på minimum Y-aksen
\item Skal forsynes med en spænding på $3,3~V$
\item Skal måle accelerationer i $\pm2~g$
%\item Skal give output i form af spænding
%\item Skal forsynes af mikrokontrolleren
\end{itemize}