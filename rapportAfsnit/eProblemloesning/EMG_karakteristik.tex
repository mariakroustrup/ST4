EMG er en målemetode, som måler elektrisk aktivitet genereret af musklerne \citep{chowdhury2013}. 
En EMG måling dækker over et samlet antal potentialer fra måleområdet, idet der aktiveres mange muskelfibre \citep{keenan2012}. \fxnote{Ved ikke om der skal skrives noget at muskelfiberne inaverers af motorneuroner, og at mængden af muskelfibre pr. motorneuron afhænger af musklen og dens funktion.}

Der anvendes normalt to forskellige typer EMG målinger, hvoraf den ene er en ikke-invasiv metode ''overflade EMG'' og den anden er en invasiv metode ''intramuskulær EMG'' \citep{chowdhury2013, keenan2012}.
I dette projekt tages der udgangspunkt i overfalde EMG målinger, hvortil der placeres elektroder på hudoverfladen. 
Ved generel anvendelse af EMG målinger, benyttes frekvensområdet ved $10-500~Hz$, da dette indlejere de mest betydende EMG-signaler. \citep{morre2003, keenan2012}.  

Denne metode kan dog blive påvirket af flere artefaktor, som bevægelses påvirkning, og støjpåvirkning fra elnettet ($50~Hz$) \citep{keenan2012}.
Ligeledes kan der ved EMG målinger fremkommer elektrisk støjpåvirkning fra omkringliggende muskeler i forhold området der måles på. Dette betegnes som crosstalk \citep{keenan2012}. \fxnote{Er det relevant at inddrage disse støj faktorer??? Kan evt. være relevat ift pilotforsøget eller diskussionen}
