\subsection{Vinkelberegning}
Beregningen af vinkler har til formål at udregne, hvor meget knæets vinkel  har ændret sig under udførsel af en squat-øvelse. Vinklen af knæet udgøres af de to accelerometre målinger, hvor disse er lagt sammen for at få den samlede vinkel over knæet.
Ifølge \autoref{sec:knaeled_squat} udføres en squat mellem $0$ til $90^{\circ}$, dette svarer til at knæet befinder sig mellem $180$ og $90^{\circ}$. Det er på denne måde muligt, at vurdere hvordan knæets vinkel har ændret sig ved udførsel af en squat-øvelse. 
For at sikre at knæet befinder sig inden for intervallet $180$ til $90^{\circ}$ under squat-øvelsen indikeres dette ved en LED, hvis intervallet overskrides. Yderligere ønskes det at outputtet ændres, hvis knæets vinkel er  over $180^{\circ}$.

 
\vspace{3mm}
\textbf{Krav:}
\begin{itemize}
\item Skal kunne udsende ét signal som repræsenterer en given vinkel
\item Skal kunne måle knæets vinkel mellem $180^{\circ}$ og $90^{\circ}$
\begin{itemize}
\item En grøn LED skal lyse når knæets vinkel befinder sig inden for dette interval
\end{itemize}
\item Skal indikere hvornår knæets vinkel er over $180^{\circ}$ og under $90^{\circ}$
\begin{itemize}
\item En rød LED skal lyse hvis knæets vinkel er over $180^{\circ}$ eller under $90^{\circ}$
\end{itemize}
\item Skal indikere hvornår knæet overstrækkes, hvilket svarer til $180^{\circ}$
\begin{itemize}
\item Hvis vinkel overstiger $180^{\circ}$ skal dette indikeres som et output på $-200^{\circ}$ for hvert accelerometer
\end{itemize}
\end{itemize}