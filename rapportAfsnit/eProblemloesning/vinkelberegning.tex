\subsection{Accelerometeralgoritme}
Accelerometeralgoritmen har til formål at omregne accelerometrenes spændinger til en vinkel. 
Hvert accelerometer indstilles til at måle en vinkel mellem 0 og $90^{\circ}$, hvorfor de tilsammen kan måle en vinkel mellem 0 og $180^{\circ}$. 
Det er hertil muligt at bestemme vinklen over knæet ved placering af accelerometrene parallelt med femur og parallelt med tibia. 
Ifølge \autoref{sec:knaeled_squat} udføres en squat mellem $0$ og $90^{\circ}$, hvilket svarer til, at knæet befinder sig mellem $90$ og $180^{\circ}$. 
På baggrund af dette anses intervallet herimellem væsentligt for udførelse af en squat-øvelse. 
Dertil visualiseres en vinkel, der befinder sig indenfor intervallet på $90-180^{\circ}$ med en grøn LED, hvor en overskridelse af intervallet indikeres med en rød LED. 
Derudover vil en overskridelse af hvert accelerometer visualisere et output på $-200^{\circ}$, hvortil en overskridelse af begge accelerometre vil visualiseres som et output på $-400^{\circ}$, hvorved EMG-algoritmen ikke påbegyndes. EMG-algoritmen fremgår af \autoref{sec:krav_emg_algo}.
 
\vspace{3mm}
\textbf{Krav:}
\begin{itemize}
%\item Skal kunne udregne en samlet vinkel over knæet
\item Skal kunne udregne knæets vinkel indenfor intervallet $90-180^{\circ}$
\begin{itemize}
\item Dette skal indikeres ved en grøn LED
\end{itemize}
\item Skal indikere, hvornår knæets vinkel er udenfor intervallet $90-180^{\circ}$
\begin{itemize}
\item Dette skal indikeres ved en rød LED
\item Hvis vinklen for ét accelerometer overstiger $90^{\circ}$, indikeres dette som et output på $-200^{\circ}$, hvortil det andet accelerometers vinkel lægges til
\item Hvis vinklen overstiger $90^{\circ}$ for hvert accelerometer, skal dette indikeres som et output på $-400^{\circ}$
\end{itemize}
\end{itemize}