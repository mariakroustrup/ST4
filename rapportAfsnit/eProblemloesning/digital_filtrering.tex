\subsection{Digital filtrering} \label{sec:teori_filter}
Der findes to former for digital filtrering; Infinite Impulse Response (IIR) og Finite Impulse Response (FIR). Der ses hertil både fordele og ulemper ved begge filtertyper \citep{blandford2012}.

FIR-filtre kan laves, således de har en lineær fase, og vil altid være stabile. 
FIR-filtre designes ved at benytte eksempelvis frekvenssampling eller en bestemt vindue-type, hvilket giver en overførselsfunktion. 
Denne overførselsfunktion kan herved benyttes som et digitalt filter \citep{blandford2012}. 

I modsætning til FIR-filtre, har IIR-filtre ikke en lineær fase, og kan være ustabile. 
Udover dette har IIR-filtre stejlere hældning i transitionsbåndet end et FIR-filter med samme antal koefficienter. 
Dette betyder, at filteret er mindre hukommelseskrævende og kan arbejde hurtigere \citep{blandford2012}. 
IIR-filtrets designprocedure er udledt af samme procedure, som de analoge filtre er designet efter.
IIR-filteret består af et forward og feedback FIR-filter, der omfatter a og b koefficienter. 
IIR-filtret udregnes ved anvendelse af følgende formel, der fremgår af \autoref{eq:iirfilt} \citep{francis2009}. 

\begin{equation} \label{eq:iirfilt}
	y(n)= \sum_{m=0}^{M} b_{m} \cdot x(n-m)- \sum^{N}_{m=1} a_{m} \cdot y(n-m)
\end{equation}

\noindent
På baggrund af den samme procedure laves IIR-filtre, ligesom analoge filtre, som Butterworth, Chebyshev type 1 og 2 og elliptiske filtre \citep{blandford2012}. Disse er illustreret på \autoref{fig:filtre}.

\begin{figure}[H]
\centering
\includegraphics[width=0.8\textwidth]{figures/filtre}
\caption{De fire filtertyper; Butterworth, Chebyshev 1 og 2 samt elliptisk \citep{wikipedia2016}.}
\label{fig:filtre}
\end{figure}

\noindent
Et Butterworthfilter, der ses på \autoref{fig:filtre}, er karakteriseret ved ikke at have nogle ripples i hverken pasbåndet eller stopbåndet. 
Hertil er der, uanset filterorden, en dæmpning på 3 dB ved knækfrekvensen \citep{nilsson2015}. Filterorden afgør dæmpning per dekade.
Et Chebyshevfilter har i modsætning til Butterworth et kortere transitionsbånd, som følge af en stejlere dæmpning, dog forekommer der ved et Chebyshevfilter enten ripples i pasbåndet eller i stopbåndet. 
Ved type 1 Chebyshevfilter, der fremgår af \autoref{fig:filtre}, ses ripples i pasbåndet samt en monotont variation i stopbåndet. 
For type 2 Chebyshev ses der ripples i stopbåndet og en monotont variation i pasbåndet \citep{nilsson2015}. 
Ved det elliptiske filter ses en stejlere dæmpning og dermed et kortere transitionsbånd end ved Butterworth- samt Chebyshevfiltre. 
Ved dette filter er der både ripples i pasbånd og stopbånd \citep{nilsson2015}. 

\vspace{3mm}
\noindent
Et moving average filter er et simpelt lavpas FIR-filter, der oftest anvendes til at udglatte et array af data, der er samplet. 
Dette er et FIR-filter, da impulsresponsen har en begrænset varighed i forhold til vinduets længde. 
Filtrets formel for udregning ses af \autoref{eq:mafl}.

\begin{equation}
	y[i]=\dfrac{1}{M}\sum^{M-1}_{j=0} x[i-j]
\label{eq:mafl}
\end{equation}

\noindent 
Et moving average filter er et vindue med en bestemt størrelse, der bevæger sig henad et array med ét element ad gangen. 
Værdien af det midterste element i vinduet vil erstattes med gennemsnitsværdien for de data, der er i hele vinduet. 
Det midterste element i vinduet må dog ikke erstattes med gennemsnitsværdien før vinduet har passeret filtret, således alle gennemsnitsværdier er baseret på de originale data. 
Af \autoref{fig:maf} fremgår et vindue, der i dette tilfælde har en størrelse på otte elementer. 
Her tages gennemsnittet af de otte elementer i vinduet, hvorefter denne værdi erstatter værdien på den fjerde plads i vinduet \citep{atmel2002}.

\begin{figure} [H]
\centering
\includegraphics[width=1\textwidth]{figures/maf}
\caption{Gennemsnitsværdien beregnes for et vindue for et moving average filter \citep{atmel2002}.}
\label{fig:maf}
\end{figure} 

\subsubsection{Filtrering af EMG-signal} \label{sec:lavpas_krav}
I pilotforsøget, \autoref{sec:pilotforsoeg}, vurderes, at udglatningen af signalet fra det analoge envelopefilter ikke er tilstrækkeligt i forhold til, at signalet skal fungere som et kontrolsystem til et exoskelet ved anvendelse af muskelaktivitet. 
Da der derfor ønskes at frafiltrere yderligere højfrekvent støj fra det forstærkede, ensrettede og lavpasfiltrerede EMG-signal, hvorfor et IIR-lavpasfilter være fordelagtigt at implementere, da et FIR-filter er mere ressourcekrævende. 
Dette digitale lavpasfilter skal fungere som endnu et envelopefilter, således signalet yderligere bliver udglattet. 
Af denne grund testes forskellige filterdesigns på resultaterne fra pilotforsøget. 
Dette indebærer, at forskellige knækfrekvenser og filterordener undersøges for at teste, hvordan disse påvirker signalet. 
På denne måde bliver det muligt at beslutte, hvordan filteret skal designes, og hvilke krav der skal opstilles. 
Filtrene designes som Butterworth-konfigurationer, da der ønskes maksimal fladhed i både pas-, transistions- og stopbåndet samt ingen ripples i pas- og stopbåndet, da dette vil påvirke signalet.

Først vælges filterets knækfrekvens ved at afprøve flere forskellige. Eksempler på disse knækfrekvenser ses på \autoref{fig:lp_knaek}, hvor $1~Hz$, $1,26~Hz$ og $3~Hz$ er repræsenteret. 

\begin{figure} [H]
\centering
\includegraphics[width=1.0\textwidth]{figures/problemloesning/lavpas_knaek.pdf}
\caption{Den blå graf viser et EMG-signal fra pilotforsøget og den røde graf viser lavpas IIR-filtre med knækfrekvenser på henholdsvis 1, 1,26 og $3~Hz$. Nederst ses udsnit af de øverste grafer, således filtrets påvirkning på signalet illustreres tydligere.}
\label{fig:lp_knaek}
\end{figure} 

\noindent
Ud fra \autoref{fig:lp_knaek} vælges en knækfrekvens på $1,26~Hz$, da filteret med denne knækfrekvens udglatter spikes og små svingninger i EMG-signalet, der vil kunne forstyrre signalet til exoskelettet. 
Samtidigt følger filteret signalet, som det kan ses på udsnittene nederst på \autoref{fig:lp_knaek}, hvilket de to andre knækfrekvenser ikke formår at gøre.

Herefter bestemmes, på samme måde som ved knækfrekvensen, hvilken filterorden der vil være mest optimal til filteret. Dette fremgår af \autoref{fig:lp_orden}.

\begin{figure} [H]
\centering
\includegraphics[width=1.0\textwidth]{figures/problemloesning/lavpas_orden.pdf}
\caption{Den blå graf viser et EMG-signal fra pilotforsøget og den røde graf viser lavpas IIR-filtre med ordner på henholdsvis 1, 2 og 3. Nederst ses udsnit af de øverste grafer, således påvirkningen af filtrets orden illustreres tydligere.}
\label{fig:lp_orden}
\end{figure} 

\noindent
Ud fra \autoref{fig:lp_orden} vælges en filterorden på 2, da denne vurderes til være tilstrækkelig. 
Der ses ingen større forskel i outputsignalet fra filteret, hvis filterordenen hæves yderligere. 
Dertil vurderes det, at en filterorden på 1 ikke er tilstrækkelig, da dette filters outputsignal afviger mest fra inputsignalet. 

\vspace{3mm}

\textbf{Krav:}
\begin{itemize}
\item Skal følge inputsignalet mest muligt  
\item Skal udformes som et Butterworth lavpasfilter
\item Skal have en knækfrekvens på $1,26~Hz$
\item Skal have en filterorden på 2
\end{itemize}

\subsubsection{Filtrering af accelerometer signaler} \label{sec:mavg_krav}
Ud fra accelerometer målingerne i pilotforsøget i \autoref{sec:pilotforsoeg}, vælges det at anvende et moving average filter. 
Dette forventes at give en mere anvendelig repræsentation af vinklen for det givne accelerometer. 
For at vælge filtrets længde afprøves forskellige længder, herunder 10, 15 og 20 samples. Dette fremgår af \autoref{fig:mavgfilter}.


\begin{figure} [H]
\centering
\includegraphics[width=1\textwidth]{figures/problemloesning/mavgfilter_matlab} 
\caption{Den blå graf viser et accelerometersignal fra pilotforsøget. Den røde graf viser et moving average filter med filterlængde på henholdsvis 10, 15 og 20 samples. I starten af grafen ses et udslag, der  svarer til filterlængden.}
\label{fig:mavgfilter}
\end{figure}

\noindent
Ud fra \autoref{fig:mavgfilter} vælges en filterlængde på 10, da det vurderes, at dette giver en acceptabel udglatning af signalet.
Herudover vurderes det at være en passende filterlængde, da det filtrerede signal repræsenterer signalerne fra pilotforsøget i højere grad sammenlignet med de andre filterlængder. 
%Ved design af et moving average filter med en filterlængde på 10 samples, går der $100~ms$ før det filtrede signal når samme amplitude som det oprindelige signal. 
%Denne forsinkelse fremgår af \autoref{fig:mavgfilter}, hvor der ses et udslag i starten, der medfører denne forsinkelse. Denne forsinkelse opstår, da filtreringen først påbegyndes efter de antal samples, der svarer til filterlængden på 10, hvilket er svarende til en forsinkelse på 10 samples à $100~ms$. 


\vspace{3mm}
\textbf{Krav:}
\begin{itemize}
\item Skal muliggøre en repræsentation af spændinger 
\item Skal have en filterlængde på $10$ samples
%\item Skal maksimal tage $100~ms$ for at opnå samme værdi som det oprindelige signal ved en konstant amplitude
%\item Skal have en maksimal forsinkelse på $100~ms$
\end{itemize}