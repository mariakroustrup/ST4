\section{Digital filtrering}

Der findes to former for digital filtrering; Infinite Impulse Response (IIR) og Finite Impulse Response (FIR). Der ses hertil både fordele og ulemper ved begge filtertyper \citep{blandford2012}.

FIR-filtre kan altid laves, således de har en lineær fase, og de er altid stabile. FIR-filtre designes ved at benytte eksempelvis frekvenssampling eller en bestemt vindue-type, hvilket giver en overførselsfunktion. Denne overførselsfunktion kan herved benyttes som digitalt filter \citep{blandford2012}. 

I modsætning til FIR-filtre, har IIR-filtre ikke en lineær fase, og de kan være ustabile. Ud over dette har IIR-filtre stejlere sidelobes end et IIR-filter med samme antal koefficienter. Dette betyder, at filteret er mindre hukommelseskrævende og kan arbejde hurtigere. IIR-filtres designprocedure er udledt af den procedure, som de analoge filtre er designet efter. Af denne grund laves IIR-filtre, ligesom analoge filtre, som Butterworth, Chebyshev type 1 og 2 og elliptiske filtre \citep{blandford2012}. 
\\

Da der ønskes at frafiltrere lavfrekvent støj fra det forstærkede, ensrettede og lavpasfiltrerede EMG-signal, vil et IIR højpasfilter være fordelagtigt for at opnå en skarpere hældning i transitionsbåndet. Herudover vil implementering af et FIR-filter kræve for meget af PSoC'en. Under pilotforsøget i \autoref{sec:pilotforsoeg} ...

