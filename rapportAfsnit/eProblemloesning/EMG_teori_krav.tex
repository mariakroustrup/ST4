\subsection{Opsamling af EMG-signaler} \label{sec:EMG_krav}
EMG er en målemetode, som måler elektrisk aktivitet genereret af muskler \citep{chowdhury2013}. 
Som tidligere nævnt i \autoref{sec:ALS} er ALS en neurodegenerativ sygdom, hvor musklen svinder ind med tiden, hvilket resulterer i mindsket muskelaktivitet. Dette påvirker EMG-målingerne, da den elektriske aktivititet hos ALS patienter derfor er mindre.

Almindeligvis kan der anvendes to former for EMG-målinger. Den ene er en ikke-invasiv metode, der betegnes overflade-EMG, og den anden er en invasiv metode, intramuskulær-EMG \citep{chowdhury2013, keenan2012}. I dette projekt anvendes overflade-EMG for at opfylde projektets overordnede krav \autoref{sec:overordnet_krav}, om at være til mindst mulig gene for patienten. Ved overflade-EMG foretages en måling over et samlet antal potentialer fra måleområdet via differensmåling, herved er det muligt at se aktiveringen af muskelfibrene \citep{keenan2012}. EMG har et frekvensområde på $10-500~Hz$, hvorfor signaler uden for dette frekvensområde, betegnes som støj \citep{morre2003, keenan2012}.  

Denne metode kan påvirkes af flere artefakter, som bevægelsespåvirkning og støjpåvirkning fra elnettet, hvilket ligger på frekvenser omkring $50~Hz$. \citep{keenan2012}.
Ligeledes kan der ved EMG-målinger fremkomme elektrisk støjpåvirkning fra omkringliggende muskler i forhold til området, der måles på. Dette betegnes som crosstalk \citep{keenan2012}. 
\vspace{3mm}

\textbf{Krav:}
\begin{itemize}
\item Skal opsamle signaler fra rectus femoris
\item Skal være anvendeligt med overflade elektroder
\item Skal opsamle muskelsignaler i frekvensområdet mellem $10-500~Hz$
\item Skal forsynes med minimum en spænding på $\pm5~V$ 
\item Skal have et justerbart gain, der ikke kan forstærke over ADC'ens arbejdsområde.
\item EMG signalet skal være en energirepræsentation.
\end{itemize}