\subsection{Emg}
EMG er en målemetode, som måler elektrisk aktivitet genereret af musklerne \citep{chowdhury2013}. 
En EMG-måling dækker over et samlet antal potentialer fra måleområdet, idet der aktiveres mange muskelfibre \citep{keenan2012}. \fxnote{Ved ikke, om der skal skrives noget om, at muskelfiberne inaverers af motorneuroner, og at mængden af muskelfibre pr. motorneuron afhænger af musklen og dens funktion = det ville være godt, inddrag lidt ALS}

Almindeligvis kan der anvendes to former for forskellige typer EMG-målinger. Den ene er en ikke-invasiv metode, der kaldes overflade-EMG, og den anden er en invasiv metode, intramuskulær EMG \citep{chowdhury2013, keenan2012}.
I dette projekt tages der udgangspunkt i overflade EMG-målinger, hvortil der placeres elektroder på hudoverfladen. 
Ved generel anvendelse af EMG-målinger, benyttes frekvensområdet ved $10-500~Hz$, hvorfor signaler uden for dette frekvensområde kan betegnes som støj \citep{morre2003, keenan2012}.  

Denne metode kan påvirkes af flere artefakter, som bevægelsespåvirkning og støjpåvirkning fra elnettet ($50~Hz$) \citep{keenan2012}.
Ligeledes kan der ved EMG-målinger fremkomme elektrisk støjpåvirkning fra omkringliggende muskler i forhold til området, der måles på. Dette betegnes som crosstalk \citep{keenan2012}. 

\textbf{Krav:}
\begin{itemize}
\item Skal opsamle signaler fra rectus femoris
\item Skal opsamle muskelsignaler i frekvensområdet mellem $10-500~Hz$
\item Skal forsynes med en spænding på $\pm5.5~V$ 
\item Skal have et justerbart gain, der ikke kan forstærke over ADC'en arbejdsområde på XX.
\item EMG signalet skal være en energirepræsentation.
\end{itemize}