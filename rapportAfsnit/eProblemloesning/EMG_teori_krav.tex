\subsection{Opsamling og behandling af EMG-signaler} \label{sec:EMG_krav}
EMG er en målemetode, som måler elektrisk aktivitet genereret af muskler \citep{chowdhury2013}. 
Som nævnt i \autoref{sec:ALS} er ALS en neurodegenerativ sygdom, hvor musklen svinder ind, hvilket resulterer i mindsket muskelaktivitet, som påvirker EMG-signalet.

Almindeligvis anvendes to former for EMG-målinger. Den ene er en ikke-invasiv metode, der betegnes overflade-EMG, og den anden er en invasiv metode, intramuskulær-EMG \citep{chowdhury2013, keenan2012}. 
I dette projekt anvendes overflade-EMG for at opfylde projektets overordnede krav, hvilket ses af \autoref{sec:overordnet_krav}, om at være til mindst mulig gene for brugeren. 
Ved overflade-EMG foretages en måling over et samlet antal potentialer fra måleområdet via differensmåling, herved er det muligt at se aktivering af muskelfibre \citep{keenan2012}. 
Studier viser, at EMG har et frekvensområde på $10-500~Hz$. 
Signaler uden for frekvensområdet, betegnes som støj \citep{morre2003, keenan2012}.  

Overflade-EMG kan påvirkes af artefakter som bevægelsespåvirkning og støjpåvirkning fra elnettet, som har frekvenser omkring $50~Hz$ \citep{keenan2012}.
Ligeledes kan der ved EMG-målinger fremkomme elektrisk støjpåvirkning fra omkringliggende biologiske signaler. Dette betegnes som crosstalk \citep{keenan2012}. 


Til forsyningen af denne analoge blok vælges en spænding ud fra komponenten, der implementeres i \autoref{sec:EMG_imp}.

\vspace{3mm}
\textbf{Krav:}
\begin{itemize}
%\item Skal opsamle muskelsignal
\item Skal være anvendeligt med overflade elektroder
\item Skal opsamle muskelsignaler i frekvensområdet mellem $10$ og $500~Hz$
\item Skal forsynes med minimum en spænding på $\pm5~V$ 
\item Skal have et justerbart gain, der tilpasses den enkelte bruger af systemet
\end{itemize}