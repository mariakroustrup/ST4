\section{Analog del} \label{sec:analog_del_krav}

\begin{figure}[H]
\centering
\includegraphics[width=1\textwidth]{figures/implementering/Blokdiagram_analog.png}
\caption{Blokdiagram over hele systemet, hvor den analoge del er fremhævet.}
\label{fig:blokdiagram_analog}
\end{figure}

\noindent
I det analoge system, som er fremhævet på \autoref{fig:blokdiagram_analog}, benyttes elektroder og accelerometre til at opsamle signaler. Systemet skal være i stand til at opsamle ElektroMyoGrafi (EMG)-signaler, hvor outputtet ønskes som værende en repræsentation af energimængden i signalet. Signalet skal derfor envelopefiltreres. 
Yderligere ønskes det at kunne justere forstærkningen, for at tilpasse amplituden af EMG-signalet, og dermed gøre systemet mere alsidigt, således det kan benyttes til flere brugere. 
Den justerbare forstærkning vil ligeledes muliggøre, at ALS-patienter kan benytte systemet i takt med det progressive muskelsvind.
Systemet skal herudover være i stand til at opsamle signaler fra accelerometrene, så accelerationen kan omregnes til en vinkel over knæet.
