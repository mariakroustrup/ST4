\section{Analog del} \label{sec:analog_del_krav}
\begin{figure}[H]
\centering
\includegraphics[width=1\textwidth]{figures/implementering/Blokdiagram_analog.png}
\caption{Blokdiagrammets analoge del}
\label{fig:blokdiagram_analog}
\end{figure}

\noindent
I det analoge system, som er illustreret på \autoref{fig:blokdiagram_analog}, benyttes sensorer til at opsamle signaler, der videresender disse informationer til en computer. Systemet skal være i stand til at opsamle EMG-signaler, hvor der ønskes en repræsentation af energimængden i signalet. For, at dette opnås skal signalet envelopefiltreres. Yderligere ønskes at kunne justere forstærkningen, for at tilpasse amplituden af EMG-signalet, og dermed gøre systemet mere alsidigt, således det kan benyttes til flere personer. Den justerbare forstærkning vil muliggøre, at ALS-patienter kan benytte systemet i takt med det progressive muskelsvind.
Systemet skal ligeledes være i stand til at opsamle signal fra accelerometre, så accelerationen fra to accelerometre kan omregnes til vinklen af knæets.
