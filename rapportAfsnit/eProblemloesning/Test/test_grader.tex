\subsection{Test af omregning til grader}

For at teste, at omregningen fra spændingen fra accelerometeret til grader fungerer i praksis, bevæges vinkeltesteren, der er vist på \autoref{fig:vinkeltest}, fra $180^{\circ}$ til $70^{\circ}$. Figur \ref{fig:spaending_vinkel_test} illustrer dette i MATLAB, hvor vinklen er illustreret på den venstre Y-akse, og hvor accelerometrenes spænding er illustreret på den højre Y-akse.

\begin{figure}[H]
\centering
\includegraphics[width=0.8\textwidth]{figures/spaending_vinkel_test}
\caption{Test af vinkelberegning. Vinkel, som er den blå graf, er svarende til den samlede vinkel mellem de to accelerometre, hvor værdierne er illustreret på Y-aksen til venstre i grader. Spændingen målt for de to accelerometre måles i forhold til Y-aksen til højre og er vist ved en rød og gul graf.}
\label{fig:spaending_vinkel_test}
\end{figure}

\noindent
Det ses på \autoref{fig:spaending_vinkel_test}, at omregningen fra spænding til grader forløber som forventet. Der tages udgangspunkt i \autoref{tab:vinkelinterval_psoc} for omregningen, hvor spændingen for hvert accelerometer omregnes til grader, og disse lægges så sammen for at få den samlede vinkel. Derudover illustrerer figuren, at vinklerne virker inden for det forventede arbejdsområde på $180^{\circ}$ til $90^{\circ}$, men også kan måle vinkler der er lavere end $90^{\circ}$}.

På figuren aflæses, at systemet registrerer en vinkel på $90^{\circ}$, når  accelerometeret på femur giver en spænding på $-136~V$ og accelerometeret på tibia giver en spænding på $-145~V$. Disse spændinger kan omregnes til vinkler ved først at finde, hvad spændingen er for én grad ud fra \autoref{tab:vinkelinterval_psoc}  i dét spændingsinterval, hvori den aflæste spænding befinder sig. Det kan aflæses i tabellen, at spændingen fra accelerometeret på femur giver en vinkel mellem 30 og $50^{\circ}$.

\begin{equation}
\dfrac{-143~V-(-84~V)}{20^{\circ}}=-2,95~V
\end{equation}
Derefter findes vinklen for accelerometeret.

\begin{equation}
\dfrac{-84~V+136~V}{2,95~V}+30^{\circ}=47,627^{\circ}
\end{equation}
Ligeledes kan vinklen fra accelerometeret på tibia udregnes til $41,429^{\circ}$. Dette giver en samlet vinkel over leddet: 
%
\begin{equation}
47,627^{\circ}+41,429^{\circ}=89,056^{\circ}
\end{equation}
Dette giver en afvigelse på $-1,049~\%$ fra $90^{\circ}$. Denne afvigelse kan godtages til dette systems formål, da der ikke er nogen fare forbundet med, at systemet overskrider de opstillede grænser for vinkler med $-1,049~\%$.









