\subsubsection{Lavpasfilter}
For at undersøge hvilken betydning det filtrede signal har i forhold til det ufiltrede signal, visualiseres disse i en samlet. De indsendte signaler er fra det optagede signal fra pilotforsøget, som er beskrevet i \autoref{sec:pilotforsoeg}. Signalerne sendes til mikrokontrollen via en UART-forbindelse, hvorved den retunerede værdi modtages. De indsendte samt returnerede værdier er visualiseret i MATLAB og fremgår af \autoref{fig:lavpas_imp}

\begin{figure}[H]
\centering
\includegraphics[width=0.8\textwidth]{figures/EMG_test}
\caption{Lavpasfilter programmeret i PSoC visualiseret i MATLAB}
\label{fig:lavpas_imp}
\end{figure}

\noindent
Figuren illustrer, at inputsignalet følger det ufiltrede signal men med forsinkelse. Til måling af behandlingstiden af dette, programmeres en timer funktion i mikrokontrolleren, der retunerer behandlingstiden til MATLAB. Ud fra dette ses et delay på XX sekunder. Dette krav accepteres, da dette ikke vil få en betydning, da der optages et antal samples og på baggrund af flere samples vurderes det, hvorvidt muskelaktiviteten er stigende eller faldende, hvilket medvirker til at systemet udfører en naturlig bevægelse.


For at vurdere om filteret er dæmper nok i forhold til de opstillede krav i \autoref{sec:lavpas_krav}, udføres en sweeptest af frekvenser fra $0-15~Hz$ med en funktionsgenerator. Dette frekvensområde er valgt på baggrund af målinger fra \autoref{sec:pilotforsoeg}, hvor det fremgår, at signalet ligger mellem $0,4-10Hz$.  Da funktionsgeneratoren ikke kan indstilles til en frekvens på $0~Hz$ indstilles denne til $1~\mu~Hz$. Amplituden sættes til 1 $V_{pp}$ med et offset på $1.65~V$, som er det halve af spændingsforsyningen på $3,3~V$. De målte værdier er multipliceret med spændingsforsyningen divideret med ADC'ens arbejdsområde, for at omregne til spænding. Spændingensforsyningen er på $3,3~V$ og ADC'ens arbejdsområde er 2048. Resultatet af sweeptesten fremgår af \autoref{fig:lavps_sweep} \textbf{a)}, mens \textbf{b)} viser kurven for $V_{pp}$ gennem sweppet.

\begin{figure}[H]
\centering
\includegraphics[width=0.8\textwidth]{figures/Lavpass_test}
\caption{Lavpasfiltrering af sweeptest fra 0 til $15~Hz$. Målingen er foretaget med et inputsignal svarende til en sinusbølge med en peak-to-peak på $1~V$. Signalet samples med en frekvens på $100~Hz$. Figur \textbf{a)} viser signalet efter filtrering, mens \textbf{b)} viser $V_{pp}$ gennem en sweeptest. Signalet er opdelt i vinduer af 500 millisekunders varighed og overlapper hinanden med 50 \%.}
\label{fig:lavps_sweep}
\end{figure}

\noindent
En dæmpning på $-3~dB$ for den valgte knækfrekvens på $5~Hz$ bestemmes samt en dekade længere ude svarende til en frekvens på $15~Hz$. Da der anvendes et 2. ordens filter, skal denne frekvens dæmpes ved $-40~dB$. Outputspændingen ved disse dæmpningsfaktorer er udregnet ved \autoref{equ:daempning1} og \autoref{equ:daempning2}. 

\begin{equation} \label{equ:daempning1}
-3~dB = 20 \cdot log_{10} \cdot (\frac{V_{out}}{0,1}) \Rightarrow V_{out} = 0,07 V
\end{equation}
\begin{equation} \label{equ:daempning2}
-40~dB = 20 \cdot log_{10} \cdot (\frac{V_{out}}{0,1}) \Rightarrow V_{out} = 0,01 V
\end{equation}

\noindent
Da målingerne for $V_{pp}$ er downsamplet er det ikke muligt at aflæse nøjagtige værdier, hvorfor der tages udgangspunkt i de nærmeste værdier. Værdierne for $5~Hz$ er $0,06~V$, mens det for $15~Hz$ er $0,008~V$. Afvigelsen fremgår af \autoref{equ:afvigelse1} og \autoref{equ:afvigelse2}.


\begin{equation} \label{equ:afvigelse1}
Afviglese_{5~Hz} = \frac{0,06V-0,07V}{0,07V} = -0,14  = - 14 \%
\end{equation}
\begin{equation} \label{equ:afvigelse2}
Afvigelse_{15~Hz} = \frac{0,008V-0,01V}{0,01V} = -0,20  = -20 \%
\end{equation}

\noindent 
Afvigelserne er for $5~HZ$ 14\%, mens det for $15~Hz$ er 20\%. Dette betyder at filteret ikke dæmper signalet nok. Da målingerne for outputspændingen, målt ved de forskellige frekvenser, ikke er målt nøjagtig aflæst, grundet downsampling, er det forventet, at der er en større afvigelse fra den teoretiske udregning. På baggrund af dette godtages filteret. 
