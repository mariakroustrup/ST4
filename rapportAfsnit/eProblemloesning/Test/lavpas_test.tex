\subsection{Lavpasfilter}
For at teste om lavpasfilteret fungerer efter kravene stillet i \autoref{sec:lavpas_krav} visualiseres forskellen mellem det ufiltrede signal og det filtrede signal. Signalerne der sendes ind er fra det optagede signal fra pilotforsøget \autoref{sec:pilotforsoeg}. Disse sendes til mikrokontrollen via UART-forbindelse, hvorved der modtages den retunerede værdi. De sendte og returnerede værdier er visualiseret i MATLAB og fremgår af \autoref{fig:lavpas_imp}

\begin{figure}[H]
\centering
\includegraphics[width=0.8\textwidth]{figures/EMG_test}
\caption{Lavpasfilter programmeret i PSoC visualiseret i MATLAB}
\label{fig:lavpas_imp}
\end{figure}

Af figuren fremgår det at inputsignalet følger det ufiltrede signal dog med et delay. Til måling af behandlingstiden af dette, programmeres en timer funktion i mikrokontrolleren, der retunerer behandlingstiden til MATLAB. Ud fra dette ses et delay på XX sekunder. Dette krav accepteres, da dette ikke vil få en betydning, da der optages et antal samples og på baggrund af flere samples vurderes det om muskelaktiviteten er stigende eller faldende, hvilket medvirker til at systemet giver en naturlig bevægelse.


For at vurdere om filteret dæmper i forhold til de opstillede krav, udføres en sweep-test af frekvenser fra $0-15~Hz$ med en funktionsgenerator. Dette frekvensområde er valgt på baggrund af målinger fra \autoref{sec:pilotforsoeg}, hvor det fremgår at signalet ligger mellem $0,4-10Hz$.  Da funktionsgeneratoren ikke kan indstilles til en frekvens på $0~Hz$ indstilles denne til $1 \mu~Hz$. Amplituden sættes til 1 $V_{pp}$ med et offset på $1.65~V$, som er det halve af spændingsforsyningen på $3,3~V$. Værdierne er ganget med $\frac{3,3V}{2048}$ for at omregne til spænding. De $3,3~V$ svarer til spændingsforsyningen, mens de 2048 er svarede til ADC'ens arbejdsområde. Resultatet af sweeptesten fremgår af \autoref{fig:lavps_sweep} \textbf{a)}, mens \textbf{b)} viser kurven for $V_{pp}$ gennem swepp'et.

\begin{figure}[H]
\centering
\includegraphics[width=0.8\textwidth]{figures/Lavpass_test}
\caption{Lavpasfiltrering af sweeptest fra 0 til $15~Hz$. Målingen er foretaget med et inputsignal svarende til en sinusbølge med en peak-to-peak på $1~V$. Signalet samples med en frekvens på $100~Hz$. Figur \textbf{a} viser signalet efter filtrering, mens \textbf{b)} viser $V_{pp}$ gennem en sweeptest. Signalet er opdelt i vinduer af 500 milisekunders varighed og overlapper hinanden med 50 \%.}
\label{fig:lavps_sweep}
\end{figure}


Dæmpning på $-3~dB$ for den valgte knækfrekvens på $5~Hz$ bestemmes samt en decade længere ude svarende til en frekvens på $15~Hz$. Da der anvendes et 2. ordens filter skal denne frekvens dæmpes ved $-40~dB$. Outputspændingen ved disse dæmpningsfaktorer er udregnet ved \autoref{equ:daempning1} og \autoref{equ:daempning2}. 

\begin{equation} \label{equ:daempning1}
-3~dB = 20 \cdot log_{10} \cdot (\frac{V_{out}}{0,1}) \Rightarrow V_{out} = 0,07 V
\end{equation}
\begin{equation} \label{equ:daempning2}
-40~dB = 20 \cdot log_{10} \cdot (\frac{V_{out}}{0,1}) \Rightarrow V_{out} = 0,01 V
\end{equation}

\noindent
Da målingerne for $V_{pp}$ er downsamplet er det ikke muligt at aflæse nøjagtige værdier, hvorfor der tages udgangspunkt i de nærmeste værdier. Værdierne for $5~Hz$ er $0,06~V$ mens det for $15~Hz$ er $0,008~V$. Afvigelsen fremgår af \autoref{equ:afvigelse1} og \autoref{equ:afvigelse2}.


\begin{equation} \label{equ:afvigelse1}
Afviglese_{5~Hz} = \frac{0,06V-0,07V}{0,07V} = -0,14  = - 14%
\end{equation}
\begin{equation} \label{equ:afvigelse2}
Afvigelse_{15~Hz} = \frac{0,008V-0,01V}{0,01V} = -0,20  = -20%
\end{equation}

\noindent 
Afvigelserne var for $5~HZ$ 14\% mens det for $15~Hz$ er 20\%. Dette betyder at filteret ikke dæmper signalet nok. Da målingerne for outputspændingen målt ved de forskellige frekvenser ikke er målt nøjagtig grundet downsampling er det forventet, at der er en større afvigelse fra det teoretiske. På baggrund af dette godtages filteret. 
