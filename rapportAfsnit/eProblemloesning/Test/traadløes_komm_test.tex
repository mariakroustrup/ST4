\subsubsection{Trådløs kommunikation}
Den trådløse kommunikation testes for at undersøge om kravene opstillet i \autoref{•}. 


Forsyning:
Kravet på at USB-dongle skal forsynes via USB overholdes i måden den trådløsekommunikation er implementeret. Her er BLE-donglen erstattet med en alternativ modtagerenhed som beskrevet i \autoref{traadloes_komm_imp}.

%Måden hvorpå den tråløse kommunikation er implementeret overholdes kravet om at BLE donglen er forsynet via USB. 

Forsinkelse:
Forsinkelsen for den trådløse kommunikation testes ved anvendelse af en debug pin på både mikrokontrolleren og på PSoC 4200M enheden der tilsluttes computeren. Da der er



Afstanden:
Til at teste kravet for afstand, programmeres mikrokontrolleren til at transmittere en værdi der tæller op fra nul. Dette værdi transmiteres 10 gange i sekundet til en computer hvorpå de modtaget data visualiseres i programmet Realterm.
Startpunket for testen er med en afstand på $1~m$ mellem modtagerenheden på computeren og mikrokontrolleren. 
Hertil øges afstanden med $1~m$ indtil der opnåes en afstand svarende det opstillede krav på $2~m$. Yderligere fortsætes forøgelsen af afstanden op til max $4~m$, eller til der ikke længere modtages data. Dette er for at undersøge om de $2~m$ markerer grænsen for den tråløse kommunikation, og dermed definere et flexområde i forhold til det opstillede krav.  

\begin{table}[H]
\begin{tabular}{|c|c|}
\hline 
Afstand [m] & Succsesfuld transmission \\ 
\hline 
1 & Ja \\ 
\hline 
2 & Ja \\ 
\hline 
3 & Ja \\ 
\hline 
4 & Ja \\ 
\hline 
\end{tabular} 
\caption{Data over afstands test for den trådløse kommunikation. Venstre søjle oplyser afstand mellem mikrokontroller og modtagerenhed, og den højre oplyser hvorvidt transmissionen har været succsesfuld eller ej.}
\label{tab:traadloes_komm_test_afstand}
\end{table}


Kommunikation med NXT:
