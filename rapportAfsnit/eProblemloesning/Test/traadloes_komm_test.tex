\subsection{Trådløs kommunikation}
Den trådløse kommunikation testes for at undersøge, hvorvidt kravene opstillet i \autoref{sec:traadloes_komm_design} opfyldes. 

\noindent
Kravet for, at USB-donglen skal forsynes via USB opfyldes ved måden, hvorved den trådløse kommunikation er implementeret. Her er BLE-donglen erstattet med en alternativ modtagerenhed som beskrevet i \autoref{traadloes_komm_imp}, og denne tilsluttes en computer via USB, hvorfra den forsynes. 

%\noindent
%Til test af forsinkelse for transmission, defineres en debug pin på mikrokontrolleren og på modtagerenheden. Når data transmitteres sættes pin'en på mikrokontrolleren høj, samt på modtagerenheden, når data modtages. Hertil kan der via et oscilloscope aflæses en forsinkelese mellem de to pin. 

\noindent
Til test af afstand, programmeres mikrokontrolleren til at transmittere en værdi, der tæller op fra nul. Denne værdi transmitteres 10 gange i sekundet til en computer, hvorpå denne data visualiseres i programmet Realterm.
Startpunket for testen er med en afstand på $1~m$ mellem modtagerenheden på computeren og mikrokontrolleren. 
Hertil øges afstanden med $1~m$ indtil der opnås en afstand svarende til det opstillede krav på $2~m$. Yderligere fortsættes forøgelsen af afstanden op til maks $4~m$, eller til der ikke længere modtages data. Dette er for at undersøge om de $2~m$ markerer grænsen for den tråløse kommunikation, og dermed definere et flexområde i forhold til det opstillede krav. Testen udføres under forhold, hvor der er fri passage mellem mikrokontroller og modtager.  

\begin{table}[H]
\centering
\begin{tabular}{|c|c|}
\hline 
Afstand [m] & Gennemført transmission \\ 
\hline 
1 & Ja \\ 
\hline 
2 & Ja \\ 
\hline 
3 & Ja \\ 
\hline 
4 & Ja \\ 
\hline 
\end{tabular} 
\caption{Data over afstandstest for den trådløse kommunikation. Venstre søjle oplyser afstand mellem mikrokontroller og modtagerenhed. Den højre oplyser, hvorvidt transmissionen har været succsesfuld eller ej.}
\label{tab:traadloes_komm_test_afstand}
\end{table}

Ud fra \autoref{tab:traadloes_komm_test_afstand} kan der konkluderes et flexrum for den trådløse kommunikation, der er $100~\%$ af det opstillede krav. Yderligere viste testen, at systemet lagrer data i en buffer i tilfælde af, at forbindelsen blev afbrudt. I det forbindelsen blev genetableret, blev den tabte data transmitteret, således ingen data gik tabt. Det kan derfor ligeledes konkluderes, at de opstillede krav for trådløs kommunikation opfyldes.
\vspace{3mm}

\textbf{Opsummering af krav:}
\begin{itemize}
\item[\text{\sffamily \checkmark}] Mikrokontrolleren skal kommunikere trådløst med en computer
\item BLE-dongle skal forsynes via USB
\begin{itemize}
\item[\text{\sffamily \checkmark}] En anden modtagerenhed er implementeret, hvilket forsynes via USB
\end{itemize}
\item Skal have en maksimal forsinkelse på 100 ms \fxnote{skal denne forsinkelse være større eller mindre?? - HUSK at ændre i brødteksten også!}
\item Skal have en kommunikationsrækkevidde på $2~m$
\end{itemize}