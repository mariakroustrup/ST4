\subsection{EMG-algoritme}
For at undersøge, om kravene i \autoref{sec:krav_emg_algo} opfyldes, testes EMG-algoritmen ved et muskelsignal på 10 sekunder, der varierer i amplitude. Inputtet til algoritmen, det filtrerede muskelsignal, og outputtet vælges til at være henholdsvis $-10~V$ eller $10~V$, hvis muskelsignalet er faldende eller stigende. Dette illustreres ved hjælp af MATLAB på \autoref{fig:emg_algo_test}. 

\begin{figure}[H]
\centering
\includegraphics[width=0.9\textwidth]{figures/EMG_algo_test}
\caption{Den blå graf og tilhørende venstre y-akse illustrerer det filtrerede muskelsignal, der er EMG-algoritmens input, og den røde graf og tilhørende højre y-akse illustrerer, om muskelsignalet er henholdsvis faldende eller stigende ved enten at udsende et signal på $-10~V$ eller $10~V$.}
\label{fig:emg_algo_test}
\end{figure}

\noindent
Ud fra de data, der fremgår af \autoref{fig:emg_algo_test}, findes EMG-inputtets lokale minima- og maksimapunkter. Tidspunkterne for disse lokale ekstrema sammenlignes med tidspunkterne, hvor outputtet skifter i spænding ved at udregne en forsinkelse i sekunder ud fra differensen på tidspunkterne for det lokale ekstrema og skift i output. Dette fremgår af \autoref{tab:emg_algo}. 

\begin{table}[H]
\centering
\begin{tabular}{|c|c|c|}
\hline 
\textbf{EMG-ekstrema [s]} & \textbf{Output-skift [s]} & \textbf{Forsinkelse [s]}\\ 
\hline 
0,29 & 0,32 & 0,03\\ 
\hline 
0,78 & 0,80 & 0,02\\ 
\hline 
0,83 & 0,86 & 0,03\\ 
\hline 
1,00 & 1,03 & 0,03\\ 
\hline 
1,10 & 1,12 & 0,02\\ 
\hline 
1,50 & 1,52 & 0,02\\ 
\hline 
2,46 & 2,49 & 0,03\\ 
\hline 
2,67 & 2,68 & 0,01\\ 
\hline 
2,94 & 2,96 & 0,02\\ 
\hline 
3,52 & 3,54 & 0,02\\ 
\hline 
3,62 & 3,63 & 0,01\\ 
\hline 
3,96 & 3,98 & 0,02\\ 
\hline 
4,33 & 4,34 & 0,01\\ 
\hline 
4,68 & 4,70 & 0,02\\ 
\hline 
5,04 & 5,05 & 0,01\\ 
\hline 
5,50 & 5,51 & 0,01\\ 
\hline 
5,67 & 5,69 & 0,02\\ 
\hline 
5,87 & 5,89 & 0,02\\ 
\hline 
7,48 & 7,54 & 0,06\\ 
\hline 
8,49 & 8,50 & 0,01\\ 
\hline 
8,83 & 8,85 & 0,02\\ 
\hline 
8,99 & 9,01 & 0,02\\ 
\hline 
9,07 & 9,09 & 0,02\\ 
\hline 
\end{tabular} 
\caption{Tabel over ekstrema, outputskift og differensen mellem disse, der er noteret som forsinkelsen.}
\label{tab:emg_algo}
\end{table}

\noindent
I \autoref{tab:emg_algo} fremgår det, at forsinkelsen fra inputtet af muskelsignalet registreres som en ændring i output-signalet er mellem $0,01$ og $0,06~s$. Den gennemsnitlige forsinkelse er på $0,0261~s$. Denne forsinkelse kan godtages til dette, da gennemsnitligt $26,1~ms$ ikke er betydeligt i forhold til den samlede forsinkelse på $832~\mu~s$??????



 \textbf{SKRIV MERE HER}.