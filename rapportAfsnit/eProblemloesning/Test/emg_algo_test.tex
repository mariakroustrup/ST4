\subsection{EMG-algoritme}
For at undersøge, om kravene i \autoref{sec:krav_emg_algo} opfyldes, testes EMG-algoritmen ved et muskelsignal på 10 sekunder, der varierer i amplitude. Inputtet til algoritmen, det filtrerede muskelsignal, og outputtet vælges til at være henholdsvis $-10$ eller $10$, hvis muskelsignalet er faldende eller stigende. Derudover ønskes det at outputsignalet er 0, hvis der opnås en vinkel på under $90$ eller over $180^{\circ}$. Dette illustreres ved hjælp af MATLAB på \autoref{fig:emg_algo_test}. 

\begin{figure}[H]
\centering
\includegraphics[width=0.9\textwidth]{figures/EMG_algo_test}
\caption{Den blå graf og tilhørende venstre y-akse illustrerer det filtrerede muskelsignal, der er EMG-algoritmens input, og den røde graf og tilhørende højre y-akse illustrerer, om muskelsignalet er henholdsvis faldende eller stigende ved enten at udsende et signal på $-10$ eller $10$.}
\label{fig:emg_algo_test}
\end{figure}

\noindent
Ud fra de data, der fremgår af \autoref{fig:emg_algo_test}, findes EMG-inputtets lokale minima- og maksimapunkter. Tidspunkterne for disse lokale ekstrema sammenlignes med tidspunkterne, hvor outputtet skifter i spænding ved at udregne en forsinkelse i sekunder ud fra differensen på tidspunkterne for det lokale ekstrema og skift i output. Dette fremgår af \autoref{tab:emg_algo}. 

\begin{table}[H]
\centering
\begin{tabular}{|c|c|c|}
\hline 
\textbf{EMG-ekstrema [s]} & \textbf{Output[s]} & \textbf{Forsinkelse [s]}\\ 
\hline 
0,32 & 0,32 & 0,00\\ 
\hline 
0,79 & 0,80 & 0,01\\ 
\hline 
0,85 & 0,86 & 0,01\\ 
\hline 
1,02 & 1,03 & 0,01\\ 
\hline 
1,11 & 1,12 & 0,01\\ 
\hline 
1,51 & 1,52 & 0,01\\ 
\hline 
2,48 & 2,49 & 0,01\\ 
\hline 
2,67 & 2,68 & 0,01\\ 
\hline 
2,95 & 2,96 & 0,01\\ 
\hline 
3,53 & 3,54 & 0,01\\ 
\hline 
3,62 & 3,63 & 0,01\\ 
\hline 
3,97 & 3,98 & 0,01\\ 
\hline 
4,33 & 4,34 & 0,01\\ 
\hline 
4,69 & 4,70 & 0,01\\ 
\hline 
5,04 & 5,05 & 0,01\\ 
\hline 
5,50 & 5,51 & 0,01\\ 
\hline 
5,68 & 5,69 & 0,01\\ 
\hline 
5,88 & 5,89 & 0,01\\ 
\hline 
7,53 & 7,54 & 0,01\\ 
\hline 
8,49 & 8,50 & 0,01\\ 
\hline 
8,84 & 8,85 & 0,01\\ 
\hline 
9,00 & 9,01 & 0,01\\ 
\hline 
9,08 & 9,09 & 0,01\\ 
\hline 
\end{tabular} 
\caption{Tabel over ekstrema, outputskift og differensen mellem disse, der er noteret som forsinkelsen.}
\label{tab:emg_algo}
\end{table}

\noindent
I \autoref{tab:emg_algo} fremgår det, at forsinkelsen fra inputtet af muskelsignalet, der registreres som en ændring i output-signalet, er mellem $0,00$ og $0,01~s$. Den gennemsnitlige forsinkelse er dermed på $9,57~ms$. Målingen, der giver en forsinkelse på $0,00~s$, kan dog ikke passe, da EMG-algoritmen bruger én sample på at skifte output fra $10$ til $-10$ eller fra $-10$ til $10$. Ingen forsinkelse vil derfor betyde, at EMG-algoritmens output begynder at ændre sig fra $-10$ til $10$, før EMG-signalet er begyndt at stige. 

Forsinkelsen på $9,57~ms$ kan forklares af, at testen er udført ved at sample EMG-algoritmens input og output ved $100~Hz$. Af denne grund er der ikke mere end 100 samples per sekund, hvilket betyder, at der er $10~ms$ mellem hver sample. Forsinkelsen er derfor sandsynligvis mindre end $9,57~ms$, men det kan ikke lade sig gøre at måle mindre tal med den nuværende samplerate. 

\vspace{3mm}
\textbf{Opsummering af krav:}
\begin{itemize}
\item Skal kunne udsende ét signal, når muskelsignalet er faldende og et andet, når det er stigende
\item Skal kunne detektere, om muskelaktiviteten er faldende eller stigende mellem to samples med 0,01 sekunders mellemrum
\begin{itemize}
\item Ved stigende muskelaktivitet skal dette indikeres som et outputsignal på $+10$
\item Ved faldende muskelaktivitet skal dette indikeres som et outputsignal på $-10$
\end{itemize}
\item Skal kunne indikere, hvis vinklen befinder sig over $180$ og under $90^{\circ}$
\begin{itemize}
\item Dette skal indikeres ved, at outputsignalet går i $0$
\end{itemize}
\end{itemize}