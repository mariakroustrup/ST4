Samplingssfrekvens for ADC'en 
Dette testes ved anvendelse af en funktionsgenerator, hvor der ud fra nyquist teori vil genereres et signus signal med en given frekvens. 
Da ADC'en opsættes til at sample med $100~Hz$, vil funktionsgeneratoren blive sat til $10~Hz$, hvor til der forventes at der konverteres 10 samples pr. sinusperiode. Dette vil blive gjort gentagende gang, hvor det vil blive undersøget hvor nøgagtigt de givende samples rammer det samme punkt. Der skal her gøres opmærksom på funktionsgeneratorens egenskab til at bevare en konstant frekvens. Disse data samples og visualiseres i matlab.  

Fejlfaktor:
Hvad for en funktionsgenerator benyttes der, og hvad siger databaldet om den evne til at vedligeholde en given frekvens?. 

ADC'ens arbejdsområde:
Dette testes ved at påføre en spænding på ADC'en indgang, hvortil det undersøges om signalet går i mætning eller ej. ADC'ens arbejdsområde er bestemt af opsætningen på mikrokontrolleren.  