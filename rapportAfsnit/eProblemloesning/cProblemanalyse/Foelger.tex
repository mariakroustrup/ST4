\subsection{Følger}

Sygdomsforløbet for ALS vil variere fra patient til patient, men der kan være fællestræk for sygdommens progression for nogle af patienterne. Sygdommen kan inddeles i 3 stadier; et tidligt stadie, et midter stadie og et endeligt stadie. I det tidlige stadie er der risiko for, at patienterne kan ignorere symptomerne, og de diagnosticeres derfor oftest først efter dette stadie.\citep{themusculardystrophyassociation2016} Disse symptomer kan være milde og kun påvirke mindre dele af kroppen, hvor musklerne eksempelvis kan være svage eller stive. Dette vil ligeledes have påvirkning på patientens balance. I det midterste stadie vil symptomerne begynde at udbrede sig. Nogle muskler paralyseres, mens andre forbliver upåvirkede. Andre muskler vil blive svagere med tiden, og dette vil blandt andet medføre problemer med synkning og vejrtrækningen. I det endelige stadie vil de fleste voluntære muskler være paralyserede, og det vil derfor ikke være muligt at indtage føde eller væske. Herudover vil det oftest i dette stadie ikke være muligt at trække vejret selv grundet respirationssvigt. \citep{themusculardystrophyassociation2016} Andre patienter kan eksempelvis opleve respirationssvigt før tabt muskelfunktion i benene, hvorfor progressionen for sygdommen ikke kan generaliseres for samtlige ALS-patienter. Til at starte med kan mindre symptomer som besvær ved at gå op ad trapper opstå. Ligeledes kan patienterne være påvirket af dropfod, når de går. Herefter vil musklerne gradvist blive svagere, og med tiden vil patienterne ikke længere være i stand til at gå.\citep{tidy2015} 


