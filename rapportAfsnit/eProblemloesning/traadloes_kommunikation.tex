\subsection{Trådløs kommunikation}\label{sec:traadloes_komm_design}
For at kommunikere trådløst benyttes Cypress BLE modul. Kommunikationstypen BLE \citep{cypressguide2014} er en energi-effektiv variation af Bluetooth-teknologi. 
Bluetooth er en standard for kortdistance trådløs teknologi, som muliggør kommunikation mellem flere enheder via radiobølger. 
Dette betyder, at systemet anvender mindre strøm på BLE-kommunikationen end på almindelig Bluetooth \citep{gupta2013}. 


Til det endelige system benyttes der en BLE-dongle. 
Dette gøres for at tillade trådløs kommunikation mellem en computer og mikrokontrolleren. En illustration kan ses på \autoref{fig:BLE_to_BLE_Dongle}. 

\begin{figure}[H]
	\centering
	\includegraphics[width=1\textwidth]{figures/BLEToBLEdongle}
	\caption{Illustration af kommunikation mellem mikrokontroller og BLE-dongle \citep{cypresspsoc2015, cypressguide2014}.}
	\label{fig:BLE_to_BLE_Dongle}
\end{figure}

\noindent
Dette tillader således trådløs test, visualisering, og debugging af mikrokontrolleren. 
BLE-donglen forsynes via USB-porten på den givne computer med $5~V$ \citep{cypressguide2014}. 
Denne form for BLE-kommunikation anvendes for at kommunikere trådløst med en computer, således en visualisering er mulig. 
For at systemet senere skal kunne anvendes til ALS-patienter under gang, skal der tages højde for en maksimal forsinkelse, for at systemet kan følge almindelig gang. 
%ALS-patienter har en gennemsnitlig gangfunktion på $1,02~m/s$ \citep{hausdorff2000}, hvorfor en forsinkelse på $100~ms$ vurderes at være acceptabel. 
Da systemet skal placeres på benet vurderes det, at en kommunikationsrækkevidde på $2~m$ er tilstrækkeligt.
\\

\textbf{Krav:}
\begin{itemize}
\item Mikrokontrolleren skal kommunikere trådløst med en computer
\item BLE-dongle skal forsynes via USB
%\item Skal have en maksimal forsinkelse på 100 ms \fxnote{skal denne forsinkelse være større eller mindre?? - HUSK at ændre i brødteksten også!}
\item Skal have en kommunikationsrækkevidde på $2~m$
\end{itemize}