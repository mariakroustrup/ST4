\section{Ændringer ved det samlede system} \label{sec:samlet_system}
Under implementering af det samlede system er det nødvendigt at ændre flere parametre for at kunne opfylde de overordnede krav, der er beskrevet i \autoref{sec:overordnet_krav}. 

\subsection{Spændingsforsyning}
Det implementeres, at systemet skal være batteridrevet, ved at mikrokontrollen og EMG-forstærkeren får spænding fra spændingensregulatoreren, der leverer en spænding på henholdsvis $5,5~V$ og $\pm 5,5~V$. 
Mikrokontrolleren er koblet til gumsticken og to accelerometre og forsyner disse med en spænding på $3,3~V$. 
Opsætningen af dette er illustreret på \autoref{fig:samlet_spaending_imp}. 

\begin{figure}[H]
\centering
\includegraphics[width=0.8\textwidth]{figures/samlet_spaending}
\caption{Illustration af koblingen af spænding til de enkelte komponenter.}
\label{fig:samlet_spaending_imp}
\end{figure}

\noindent
Mikrokontrolleren testes yderligere, da systemet ønskes at være batteridrevet, hvorfor mikrokontrolleren ikke kan forsynes via USB. 
Derfor testes, hvor meget spænding mikrokontrollerens $3,3~V$'s forsyningspin leverer, idet mikrokontrolleren forsynes via henholdsvis USB og spændingsregulatoren. 
Ved forsyning via USB, blev en spænding målt til $3,273~V$, og via spændingsregulatoren blev en spænding målt til $3,278~V$.

\subsection{Accelerometeralgoritme}
På baggrund af accelerometrenes fastgørelse på vinkeltesteren, der ses af \autoref{fig:vinkeltest}, har været uhensigtsmæssig, fortages nye målinger. 
Hertil er vinkeltesteren blevet optimeret, således accelerometrene fastgørelse er mere stabil.
  
Offset for accelerometrene ændres ligeledes. Dette er målt til $1,5904~V$ og $1,5598~V$ for placering af accelerometre parallelt med henholdsvis femur og tibia. 
Det nye konverterede output svarende til en vinkel fremgår af \autoref{tab:samlet_vinkel_imp}. 

\begin{table}[H]
\centering
\begin{tabular}{|c|c|c|}
\hline
\multicolumn{1}{|l|}{\textbf{Vinkel {[}$^{\circ}${]}}} & \textbf{\begin{tabular}[c]{@{}c@{}}Konverterede output fra \\ accelerometer placeret \\ parallelt med femur \end{tabular}} & \textbf{\begin{tabular}[c]{@{}c@{}}Konverterede output fra \\ accelerometer placeret \\ parallelt med tibia\end{tabular}} \\ \hline
\textbf{0}                                                      & -195                                                                               & -170                                                                                      \\ \hline
\textbf{10}                                                     & -191                                                                             & -164                                                                                     \\ \hline
\textbf{30}                                                     & -165                                                                             & -142                                                                                    \\ \hline
\textbf{50}                                                     & -122                                                                            & -101                                                                                   \\ \hline
\textbf{70}                                                     & -60
& -42                                                                                   \\ \hline
\textbf{80}                                                     & -25	
& -4                                                                                   \\ \hline
\textbf{90}                                                     & 0                                                                            & 0                                                                                   \\ \hline
\end{tabular}
\caption{Konverterede outputs fra accelerometre placeret parallelt med henholdsvis femur og tibia svarende til en given vinkel.}
\label{tab:samlet_vinkel_imp}
\end{table}

