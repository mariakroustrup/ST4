\subsubsection{Low power mode}
Det kan være fordelagtigt at anvende sleep mode, ved opsætning ADC'en eller digital kommunikation. Dette er situation hvor andre perifer enheder skal forblive aktive, men uden CPU'ens aktivitet er nødvendig. På denne måde vil det være muligt at reducere strømforbruget mellem A/D-konvertinger samt transaktioner under den digitale kommunikation.

Der findes forskellige former af low power mode. Herunder sleep mode, deep-sleep mode, hibernate mode og stop mode. Systemet anvender sleep mode, hvor kun CPU'en er slukket, mens alle systemets andre enheder er aktive. Systemet befinder sig i sleep mode størstedelen af tiden, men dette bliver afbrudt, når der forekommer et interrupt.  \citep{cypresspsoc420152}
