\subsection{Low power mode}

Der findes forskellige grader af low power mode. Herunder sleep mode, deep-sleep mode, hibernate mode og stop mode. Systemet anvender sleep mode, hvor kun CPU'en er slukket, mens alle systemets andre enheder er aktive. Systemet befinder sig i sleep mode størstedelen af tiden, men dette bliver afbrudt, når der forekommer et interrupt. Et hvert interrupt kan anvendes til at vække systemet fra sleep mode, herefter vil systemet gå ud af sin sleep mode, når der er udløst et interrupt. 

Det kan være fordelagtigt at anvende sleep mode, når eksempelvis ADC'en eller digital kommunikation, hvor andre peripherals skal forblive aktive, men uden CPU'ens aktivitet er nødvendig. På denne måde vil det være muligt at reducere strømforbruget mellem AD-konvertinger samt transaktioner under den digitale kommunikation.\citep{cypresspsoc420152}
