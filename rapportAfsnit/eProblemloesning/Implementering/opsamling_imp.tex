\section{Analog del}
\begin{figure}[H]
\centering
\includegraphics[width=1\textwidth]{figures/implementering/Blokdiagram_analog.png}
\caption{Blokdiagrammets analoge del, der implementeres i det følgende afsnit.}
\label{fig:blokdiagram_analog1}
\end{figure}

\noindent
Til implementering af det analoge system, som er illustreret på \autoref{fig:blokdiagram_analog1}, bestemmes der ud fra de opstillede krav i afsnit \autoref{sec:analog_del_krav}, at der skal indgå EMG-signaler og accelerometre til signaloopsamling og behandling. For uden at leve op til de opstillede krav, var disse komponenter til rådighed. Til at forsyne EMG-signaler er der anvendt en spændingsforsyning, som ligeledes er en udleveret komponenten som opfylder kravene stillet i \autoref{sec:krav_spaending}.


%Til implementering af det analoge system, som er illustreret på \autoref{fig:blokdiagram_analog1}, bestemmes der ud fra de opstillede krav i \autoref{sec:analog_del_krav}, at der skal indgå en EMG-forstærker og accelerometre, der introduceres i \autoref{sec:EMG_krav} og \autoref{sec:acc_teori} i implementeringen af systemet. På baggrund af de overordnet krav til systemet er der valgt at anvende EMG og accelerometre, hvorudfra disse krav samt de opstillede krav har medvirket til tilvalg og fravalg af sensorer. Derudover har valget af EMG-forstærker og accelerometre til opsamling af EMG-signal og vinkler været afhængige af, hvilke komponenter der er til rådighed, som opfylder de opstillede krav. 




