\section{Flowdigram}
I dette afsnit fremgår implementeringen af systemets blokke \autoref{sec:blokdiagram}. Flowdiagrammer er anvendt for at visualisere opbygningen og sammenhængen mellem blokkene. Flowdiagrammene er opdelt og består af et overordnet, som beskriver hele processen og et initialiserende samt et, der viser EMG-algoritmen. Udover den visualiserende del vil det blive uddybet, hvilke funktioner de enkelte figurer indeholder. Anvendelsen af de forskellige figurer er beskrevet i \autoref{sec:flowhaandtering}.

\subsection{Overordnet flowdiagram}	
Det analoge signal, som optages af de implemterende sensorer skal konverteres fra analogt til digitalt, hvorved det efterfølgende kan implementeres i softwaren. Det overordnede flowdiagram fremgår af \autoref{fig:overordnet_flow}. For at implementering af softwaren vil foreløbe skal der ske et interrupt, som igangsætter de efterfølgende funktioner. Dette igangsætter et timer interrupt, der tæller inden for XX interval. Herefter sker der en intialisering af systemet.

\begin{figure}[H]
\centering
\includegraphics[width=0.6\textwidth]{figures/implementering/overordnet_flow.png}
\caption{Overordnet flowdigram som viser opbyggelsen af det samlede system}
\label{fig:overordnet_flow}
\end{figure}


\subsection{Initialiserende flowdiagram}
I intialiseringsprocessen, der fremgår af \autoref{fig:initialiserende_flow} sker opsætningen af ADC'en, hvorefter det er muligt at se, hvorvidt kravet for det analoge input opfyldes. Ved et analogt input sker en A/D-konvertering, hvor det analoge signal digitaliseres. For at kunne behandle dataen og kommunikere trådløst igangsættes et setup, hvor BLE tilkobes og signalet offsetjusteres. Hvis kravet for det analoge input ikke opfyldes, vil systemet gå i low power mode. Efter setup skal det vurderes, hvorvidt inputtet befinder sig inden for en spænding svarende til 0 til 90 graders vinkel, hvis dette er tilfældet vil signalet starte EMG-algortimen. Hvis inputsignalet ikke er tilstrækkeligt, vil en rød LED lyse og intialiseringsprocessen vil starte på ny. 
\begin{figure}[H]
\centering
\includegraphics[width=0.6\textwidth]{figures/implementering/initialiserende_flow.png}
\caption{Initialiserende flowdigram som viser opbyggelsen af den intialiserende del af systemet}
\label{fig:initialiserende_flow}
\end{figure}

\subsection{EMG-algoritme}
EMG-algoritmen fremgår af \autoref{fig:Emg_algo}. Hvis inputtet svarer til en spændingen mellem 0 til 90 grader vurderes det, hvorvidt muskelaktiviteten er faldende eller stigende. Hvis der ingen muskelaktivitet er, påbegyndes en loop, hvor det igen vurderes om muskelaktiviteten er stigende eller faldende. Hvis der efter 30 sekunder fortsat ingen muskelaktivitet er, afsluttes EMG-algoritmen. Hvis muskelaktiviteten er stigende, vil vinklen blive større, og herved slutter EMG-algoritmen og sender et input videre til BLE, hvorved en fleksion af knæleddet påbegyndes. Hvis muskelaktiviteten er faldende vil viklen blive mindre, hvilket ligeledes slutter EMG-algoritmen og sender et input videre til BLE, hvorved en ektension af knæleddet påbegyndes. 
\begin{figure}[H]
\centering
\includegraphics[width=0.6\textwidth]{figures/implementering/EMG_algo.png}
\caption{EMG-algoritmen viser opbyggelsen af EMG-algortimens del af systemet}
\label{fig:Emg_algo}
\end{figure}

