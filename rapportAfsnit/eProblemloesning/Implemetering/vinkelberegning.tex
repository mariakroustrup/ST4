\subsection{Beregning af vinkler på PSoC}
I \autoref{sec:test_acc} fremgår det at der er linær sammenhæng mellem vinklerne over tid, hvorfor der kan udføres en lineær interpolation over dataen fra målinger af accelerometrene i forskellige vinkler. På denne måde er det muligt at bestemme en hvilken som helst vinkel for en tilsvarende  spænding. De målte spændinger opdeles i 10 intervaller som fremgår i \autoref{tab:vinkelinterval}. 

\begin{table}[H]
	\centering
	\begin{tabular}{|l|l|l|}
				& \textit{Accelerometer placeret på låret} &				
	\textbf{Interval} & \textbf{Målt spænding} & \textbf{Målt spænding} 		\\ \hline	
    \textbf{0-20} 			& $0~V$							& $-0,1973~V$    \\ \hline
    \textbf{20-40} 			& $-0,1973~V$					& $-0,0661~V$	\\ \hline
    \textbf{40-60} 			& $-0,0661~V$					& $-0,1381~V$	\\ \hline
    \textbf{60-80} 			& $-0,1381~V$					& $-0,2292~V$	\\ \hline
    \textbf{80-90} 			& $-0,2292~V$					& $-0.2825~V$	\\ \hline
    				& \textit{Accelerometer placeret på skinnebenet} &		
    	\textbf{Interval} & \textbf{Målt spænding} & \textbf{Målt spænding} 		\\ \hline	
    \textbf{90-80}			& $0.3142~V$ 					& $0,2291~V$	    \\ \hline
    \textbf{80-60}			& $0,2291~V$						& $0,1388~V$	 	\\ \hline
    \textbf{60-40}			& $0,1388~V$						& $0,0648~V$		\\ \hline
    \textbf{40-20}			& $0,0648~V$						& $0,0177~V$		\\ \hline
    \textbf{20-0}			& $0,0177~V$						& $0~V$			\\ \hline
	\end{tabular}
	\caption{Spændingen målt i vinklerne fra 0 til 90$^{circ}$ for accelerometrene placeret på både låret og skinnebenet. Den første spænding svarer til start intervallet og den sidste til der hvor intervallet stopper.}
	\label{tab:vinkelinterval}
\end{table}

Den målte spænding er kun vist med fire decimaler, men er oprindeligt med 15 decimaler. Spændingen er målt i vinklerne fra 0 til 90$^{circ}$ for accelerometrene placeret på både låret og skinnebenet. Den første målte spænding er for startværdien i intervallet og den sidste spænding er slutværdien i intervallet.

Ud fra de målte værdier i de forskellige intervallet er der opstillet i en funktion indehold ifelse-løkker, hvorved det er muligt at vurdere hvilket interval en given spænding befinder sig i. I hver enkelt løkke bliver der anvendt lineær polation, som har til opgave at finde en vinkel, der er svarende til en spænding som ligger mellem intervallet og returnere denne. 

Når der er lavet lineær interpolation på dataen skal de målte data fra låret og fra skinnebenet ligges sammen for at få den samlede vinkel af hvor langt personen har bevæget sig, og derved befinder sig i squat-øvelsen. 



