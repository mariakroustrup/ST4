% !TeX spellcheck = da_DK
\section{Problemafgrænsning}
I dette projekt fokuseres på ALS-patienter samt muligheden for opretholdelse af kropsfunktioner ved benyttelse af body augmentation-hjælpemidler. 

Da ALS-patienter oplever progressivt muskelsvind, har dette indflydelse på deres selvstændighed, da de gradvist mister kontrollen over deres legemsdele. Da der kun eksisterer palliative behandlinger til ALS-patienter fokuseres der på at afhjælpe deres fysiske mangler ved brug af et exoskelet som aflastning. Dette system kræver minimal fysisk indsats at anvende. 
Ved opretholdes af de fysiske funktioner vil dette ligeledes have en positiv effekt på den sundhedsrelaterede livskvalitet, da det vil kunne resultere i en større selvstændighed. Der er dog ingen garanti for, at det vil gavne den overordnede livskvalitet.  

Idet ALS vil resultere i, at patienten mister evnen til at kunne gå, fokuseres der på at opretholde denne funktion. Til dette er det fremhævet, at knæet er det vigtigste led i forhold til gang. Dertil tages udgangspunkt i knæleddet, hvor musklerne omkringliggende knæet afhjælpes ved anvendelse af et exoskelet.

\subsection{Problemformulering}
Hvordan kan body augmentation anvendes for at aflaste ALS-patienters lårmuskulatur under en squat-bevægelse?