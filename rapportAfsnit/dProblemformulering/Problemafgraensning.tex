% !TeX spellcheck = da_DK
\subsection{Problemafgrænsning}
I dette projekt fokuseres der på ALS patienter samt muligheden for opretholdelse af kropsfunktioner ved benyttelse af body augmentations hjælpemidler. 

Da ALS patienter oplever progressiv svind af deres muskler, har dette indflydelse på deres selvstændighed, da de bl.a. gradvist mister kontrollen over deres legemesdele. Da der kun eksisterer palliative behandlinger til ALS patienter fokuseres der på at afhjælpe deres fysiske mangel ved brug af et exoskelet som aflastning. Dette system kræver minimal fysisk indstat at anvende, hvilket er passende til den typiske ALS patient. 
Ved opretholdes af de fysiske funktioner vil dette ligeledes have en positiv effekt på den sundhedsrelateret livskvalitet, da dette vil kunne resultere i en større selvstændighed. Der er dog igen garanti på at det vil gavne den overordnede livskvalitet hos ALS patienten.  

Idet ALS vil resultere i, at patienten mister evnen til at kunne gå, fokuseres der på at opretholde denne funktion. Til dette er det fremhævet, at knæet er det vigtigste led i forhold til at kunne gå. Dertil vælges der at tages udgangspunkt i knæleddet, hvor musklerne omkringliggende knæet afhjælpes ved anvendelse af et exoskelet.

\subsection{Problemformulering}
Hvordan kan et exoskelet anvendes over et knæled, med henblik på, at ALS patienter skal opretholde deres evne til at gå 