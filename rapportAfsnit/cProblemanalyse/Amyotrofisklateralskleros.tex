\chapter{Amyotrofisk lateral sklerose}
%Tilføj billede til afsnittet (billede af motorneuroner).
ALS er en neurodegenerativ sygdom, der resulterer i muskelsvaghed. Ved ALS degereneres motorneuronerne i hjernen samt rygsøjlen i takt med sygdommens fremskreden. De første symptomer herpå er kramper, svaghed samt stive muskler, det kan begynde som muskelsvaghed i arme eller ben, talebesvær eller svaghed i de muskler, som styrer respirationen. Symptomer, der begynder i arme eller ben kaldes 'limb onset ALS', mens talebesvær samt synkebesvær refereres til 'bulbar onset ALS'. 
ALS opleves individuelt, hvorved nogle ALS patienter først oplever muskelsvaghed i deres ben, mens andre oplever muskelsvaghed i deres hænder og arme eller besvær i form af tale- eller synkebesvær. \citep{miller2005} \citep{nationalinstitute2016}
 
Grunden til svagheden er, at musklerne indikerer abnormiteter i de lavere motorneuroner. De lavere motorneuroner er de nerveceller, der bærer information fra rygmarven til musklerne. Symptomer på abnormiteter i de lavere motorneuroner ses som muskelsvaghed samt muskelkramper og atrofi.
Ligeledes kan de øvre motorneuroner påvirkes, disse motorneuroner bærer kommunikationen mellem hjernen og de nedre motorneuroner i rygmarven. Dette medfører, at beskeden fra hjernen har komplikationer til at komme ned til det givne sted. Dette ses som spasticitet samt overdrevne reflekser.\citep{nationalinstitute2016}
Årsagen til, at ALS opstår er oftes ukendt, dog ses en arvelighed i 5-10 \% af tilfældene. Herudaf anses 20 \% til at have det muteret Superocide dismutase 1 (SOD-1 gen), hvilket resulterer i tab af motorneuroner. \citep{miller2005}

Påtrods af, at ALS opleves individuelt både i forhold til progressionen af sygdommen samt, hvilke komplikationer de oplever, vil flere patienter i sidste ende opleve besvær ved at stå og gå samt benytte deres hænder og arme. Herudover opleves synke- og tyggebesvær, hvilket forringer deres evne til at spise normalt. I de seneste stadier af ALS mister patienter even til selv at trække vejret og bliver derfor afhængig af ventilationsstøtte. Den mest almindelig dødsårsag er respirationssvigt, hvilket oftes sker før 3 år efter diagnosen er stillet. 25 \% har en overlevelsesrate på 5 år, og kun 10 \% lever længere end 10 år efter diagnosen er stillet. \citep{grehl2011} \citep{miller2005}

