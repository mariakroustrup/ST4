\chapter{Amyotrofisk lateral sklerose}
% Mere til afsnittet: Progressionen af sygdommen (distalt til proksimalt? hastighed), find noget om hvor mange og hvornår man kan miste evnen til at gå (hvis det er meget individuelt, så skriv det), evt. sammenhænge mellem hvor sygdommen først rammer, og senere progression. Tilføj billede til afsnittet (billede af motorneuroner).
ALS er en neurodegenerativ sygdom, der resulterer i muskelsvaghed. Ved ALS degereneres motorneuronerne i hjernen samt rygsøjlen i takt med sygdommens fremskreden. Symptomerne herpå er udvikling af svaghed i områder af kroppen, det kan begynde i en fod, hånd, arm, ben eller muskler, der styrer respirationen. Musklerne bliver svage, langsomme eller tynde. Grunden til dette er, at musklerne indikerer abnormiteter i de lavere motorneuroner. De lavere motorneuroner er de nerveceller, der bærer information fra rygmarven til musklerne. 
Ligeledes kan de øvre motorneuroner påvirkes, disse motorneuroner bærer kommunikationen mellem hjernen og de nedre motorneuroner i rygmarven. Dette medfører, at beskeden fra hjernen har komplikationer til at komme ned til det givne sted, dette kan gå langsomt eller slet ikke fungere. 

%Hvad er det for et mønster, der er ved 5-10 % af ALS-patienter? Hvad er meningen med sammenligningerne? Det er måske ikke så relevant at forklare, hvad alzheimers og parkinsons er. 
ALS sammenlignes af flere eksperter til at være i relation med andre neurologiske sygdomme, såsom Alzheimers, hvorfor celler ansvarlig for hukommelse forringes. En anden neurologisk sygdom som ALS kan sammenlignes med er parkinson syge. Her forringes eller stopper koordineringen af kroppen ved hjælp af nerveceller helt med at fungere. Ved disse tre neurologiske sygdomme er det bestemte nerveceller, som degenerer. 
Årsagen til at ALS opstår er i de fleste tilfælde ukendte, dog ses et mønster ved 5-10 \% af mennesker med ALS. Herudad anses 20 \% til at have det muteret Superocide dismutase 1 (SOD-1 gen), hvilket resulterer i tab af motorneuroner. 

%  Vi er ikke så sikre på, at det i fxnoten er særligt nødvendigt. Vi skal være helt sikre på, at det er den sammenhæng, som du også selv skriver. 
\fxnote{Der er noget med 'overdreven stimulering' af motorisk nerceller med glutamat - sker notmalt i hjernen + rygmarven. Den her overdreven stimulering af glutamat = tab af motorneuroner. 
Der er noget med riluzol som kan forstyrre opbygningen af gultamat og forsinker så udviklingen af sygdommen. 

Inflammation i hjernen og rygmarven, ny beviser: overdreven betændelse sker ved motorstyret regioner. 

Jeg har kun en kilde på disse ting, og derfor tænker jeg, at der skal mere litteratur til, før jeg tør at skrive det.}