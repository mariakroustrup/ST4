% !TeX spellcheck = da_DK
\section{Gangfunktion}
Efterhånden som ALS-patienter mister muskelkraft, vil bevægeligheden i deres led nedsættes, eftersom de ikke har tilstrækkelig muskelkraft til at udnytte leddenes bevægelighed. Af denne grund opstår der kontrakturer i leddene, og muskelstramninger i de muskler, der er omkringliggende. Ved gang anvendes knæ-, hofte- og ankelleddet, hvilket fremgår af \autoref{fig:knaet}, og hvis disse led ikke akviteres, opstår der muskelstramninger i benenes muskler \citep{instforms2008}. Knæleddet vælges som udgangspunkt for et muligt body augmentation-system i form af et exoskelet, da knæleddet er et hængselled og derfor har et begrænset antal frihedsgrader. Knæleddet har én frihedsgrad, modsat andre mere komplekse led, hvilket gør at leddet kun kan bevæge sig i en akse. Det antages derfor at knæleddet er et af de led som er simplest at opbygge et system omkring og opsamle signaler fra de omkringliggende muskler. Hvis der kan laves et exoskelet omkring knæleddet, vil det kunne antages, at samme princip kan muliggøres ved henholdsvis hofte- og ankelleddet, hvorved gangfunktionen kan opretholdes.

\begin{figure} [H]
\centering
\includegraphics[width=0.8\textwidth]{figures/knaet}
\caption{Aktivering af hofte, knæ og ankel under gang \citep{orthopedics2016}.}
\label{fig:knaet}
\end{figure} 

\subsection{Knæets opbygning}
Knæleddet er som tidligere nævnt et hængselled, hvilket medvirker til få frihedsgrader, som gør at knæet kan rotere begrænset samt fleksere og ekstensere.
Knæet består af tre separate ledforbindelser. To, der er forbundet mellem femur og tibia, samt et mellem patella og femur, hvilket fremgår af \autoref{fig:knae_anatomi}. Ud over de tre separate ledforbindelser stabiliseres knæet af syv ledbånd. Ét af de syv ledbånd er patellarsenen, som er ansvarlig under extension af knæet. Derudover er der to ledbånd, som strækker sig mellem femur, tibia og fibia, hvilket er med til at styrke knæleddets overflade posteriort. Inde i ledkapslen befinder det forreste korsbånd (ACL) og det bagerste korsbånd (PCL), som har til opgave at fastgøre indre knoglefremspring af tibia til knoglefremspringet på femur. Korsbåndene har til opgave at begrænse anteriore og posteriore bevægelser af femur og er med til at opretholde retningen af knoglefremspringene. Det tibiale kollaterale ligament forstærker den mediale flade af knæleddet og det fibulære kollaterale ligament forstærker sidefladen. Disse ligamenter anvendes kun ved fuld ekstension \citep{martini2012}.

\begin{figure}[H]
\centering
\includegraphics[width=0.8\textwidth]{figures/knae_anatomi}
\caption{Knæets anatomiske opbygning \citep{martini2012}}
\label{fig:knae_anatomi}
\end{figure} 

\subsection{Knæets funktion}
Ved gang aktiveres quadricepsmusklerne, der sidder anteriort på femur, og hasemusklerne, der sidder poseriort på femur, hvilket fremgår af \autoref{fig:knae_anatomi}. Quadricepsmusklerne består af rectus femoris, vastus intermedius, vastus medialis og vastus lateralis. hasemusklerne består af biceps femoris, semitendinosus og semimembranosus. Ved bevægelse foretager quadriceps- eller hasemusklerne ekstension eller fleksion, hvorved de fungerer som hinandens agonister eller antagonister under bevægelse \citep{martini2012}. 

Som tidligere nævnt anvendes hofte, knæ og ankler under gang. Udover disse led er også kropsposituren og sving af leddene afgørende for gangfunktionen. Det fremgår af \autoref{fig:knaet}, hvordan de forskellige led udfører fleksion, ekstension og ændres fra ekstension til neutral bevægelse under gang \citep{martini2012}.

\subsubsection{Knæets funktion under en squat-øvelse}
% mere om, at det kun er lårets muskulatur, der benyttes under squat
Den dynamiske squat-øvelse er en udbredt træningsøvelse, som kræver et højt niveau af styrke i flere muskelregioner. Squat aktiverer primært hofte-, lår- og rygmuskulaturen, som alle er vigtige muskler under gang, løb, spring og løft. Herudover anvendes squat også som et redskab til rehabilitering af knæet, hvilket skyldes den måde, som knæet belastes under squat \citep{escamilla2001}. 

Knæets funktion for bøjningen af benet kan dermed ses ved udførelse af en squat-øvelse. En squat-øvelse udføres ved at stå i en oprejst position med knæ og hofte fuldt udstrakt. Herefter udføres en squat-øvelse i en kontinuerlig bevægelse, indtil den ønskede dybde nåes, hvorefter der udføres en kontinuerlig bevægelse tilbage til oprejst position \citep{escamilla2001}.

Squat kan udføres med varierende grader af fleksion af knæet. De mest anvendte varianter af øvelsen er halv eller fuld squat. Den halve squat-øvelse udføres indtil lårene er parallelle med jorden, hvilket svarer til en fleksion af knæet fra omkring $0-100^{\circ}$. Den dybe squat-øvelse udføres indtil det posteriore del af låret og læggen kommer i kontakt med hinanden. Den dybe squat anbefales mere trænede personer, hvorfor den halve squat typisk er foretrukket til genoptræning af knæet \citep{escamilla2001}.

Ved udførelse af en squat-øvelse er det primært lårmusklerne, quadriceps- og hasemusklerne, der aktiveres. Under denne øvelse aktiveres quadricepsmusklerne ved $80-90^{\circ}$ fleksion af knæet og aktiviteten af musklerne er herefter konsistent. Under en squat-øvelse aktiveres vastus intermedius, vastus medialis samt vastus lateris mere, da disse muskler er én ledmuskel, end rectus femoris der er en to-ledsmuskel \fxnote{hvorfor aktiveres én-ledsmuskler mere end to-ledsmuskler - jeg kan ikke finde det med hasemusklerne i den kilde der står til afsnittet?}. Hasemusklerne aktiveres ved en $45^{\circ}$ fleksion af knæleddet \citep{schoenfeld2010}. 


 %der er forbundet mellem femur, tibia og patella. Knæet har fire ledbånd. To af disse er side-ligamenterne, der sidder omkring knæleddet. De resterende to er korsbåndene, der sidder på skrå inden i knæet.
%Knæet fire ledbånd sikrer stabilisering af knæet og sørger for at knoglerne bevæger sig rigtigt.  Det er knæleddet, der gør det muligt for kroppen at kunne udføre aktiviteter som at kunne gå, løbe, og eksempelvis squatte. Ved gang aktiveres både quadriceps musklerne (rectus femoris, vastus intermedius, vastus medialis, vastus lateralis) der sidder anteriort for låret samt hamstring musklerne (biceps femoris, semitendinosus, Semimembranosus), der sidder posteriort for låret og kontraherer med quadriceps musklerne. 

