% !TeX spellcheck = da_DK
\subsection{Knæleddet}

Knæet samt knæleddet er et af kroppens vigtigste led for, at mennesket er i stand til at gå. Knæleddet er et hængselled, der er forbundet mellem femur, tibia og patella. Knæet har fire ledbånd. To af disse er side-ligamenterne, der sidder omkring knæleddet. De resterende to er korsbåndene, der sidder på skrå inden i knæet. Knæet fire ledbånd sikrer stabilisering af knæet og sørger for at knoglerne bevæger sig rigtigt.  Det er knæleddet, der gør det muligt for kroppen at kunne udføre aktiviteter som at kunne gå, løbe, og eksempelvis squatte. Ved gang aktiveres både quadriceps musklerne (rectus femoris, vastus intermedius, vastus medialis, vastus lateralis) der sidder anteriort for låret samt hamstring musklerne (biceps femoris, semitendinosus, Semimembranosus), der sidder posteriort for låret og kontrahere med quadriceps musklerne. Ved gang anvendes både hofteled, knæled og ankelled. 


\begin{figure}
\centering
\includegraphics[width=0.45]\textwidth{knaet}{figures/knaet}
\label{fig:knaet}
\end{figure} 


Ovenstående billede viser ekstension samt fleksion af hofte, knæ og ankel ved gang. Ved bøjning af knæet, når en squat øvelse udføres og knæet bøjes aktiveres quadriceps musklerne ved en 80-90 graders fleksion og er herefter konsistent, hvor hamstring muskler aktiveres ved 45 graders fleksion. Af quadriceps musklerne ses størst aktivering af vastus intermedius, vastus medialis samt vastus lateris, da disse msuskler er én ledmuskel, hvor rectus femoris er en to-ledsmuskel. 
