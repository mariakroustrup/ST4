% !TeX spellcheck = da_DK
\subsection{Nuværende teknologier og hjælpemidler}
Som tidligere nævnt er ALS en livstruende sygdom, hvor følgerne sker gradvist, hvilket gør at patienternes funktionelle evner svækkes over sigt, hvorfor der er behov for en række hjælpemidler som helt eller delvist kan være en hjælp i hverdagen. Nogle af hjælpemidlerne er i starten af sygdommen for at patienterne kan klare sig selvstændigt, hvor der senere er behov for andre hjælpemidler samt helt eller delvist hjælp fra en ægtefælle eller plejepersonale. Da der ikke findes nogle behandlinger mod ALS er behandlingen palliativ, hvor der på nuværende tidspunkt anvendes teknologiske og personlige hjælpemidler. [1]

\subsubsection{Teknologiske og personlige hjælpemidler}
De mest anvendte hjælpemidler for patienter med ALS er teknologiske som f.eks.  kørestole, toiletstole og stokke. Hjælpemidlerne er alle redskaber som støtter og aflaster patienten. Derudover anvendes der mere personlige hjælpemidler som i stedet er tilpasset patienterne individuelt. Patienterne har på denne måde et særligt behov for hjælpemidler som f.eks. tilpasset kørestole, tilpasset fodtøj og høreapparat. [1]

\subsubsection{Udfordringer og nye muligheder}
Patienter med ALS kan i starten gå uden yderligere besvær, men bliver mere og mere afhængig af hjælpemidler, hvor de til sidst ender i kørestol. På denne måde forsvinder deres selvstændighed, da de er afhængig af deres kørestol samt hjælp fra plejepersonale eller ægtefælle. [1,2] Dette giver nogle begrænsninger for patienten og medvirker til en forringet livskvalitet. En mulig måde for at give patienter med ALS nye muligheder er anvendelse af body augmentation, som er en forøgelse af kropsfunktioner.[3] 

\subsubsection{Exoskelet}
En form for body augmentation er anvendelse af exoskelet. Exoskelet  anvender den menneskelige intelligens og kombinerer den med en maskines magt. På denne måde er det muligt at maskinen fungerer som en menneskelig operatør, som kan forbedre menneskets styrke eller genoprette bevægelse. [4] Dette gør at exoskelettet kan anvendes som et hjælpemiddel til f.eks. tunge løft eller til patienter som lider af et handicap eller form for skader, hvorved det gør det muligt at aflaste patienten. [5] Forsøg har påvist at det er muligt at gå ved brug af exoskelet ved patienter som er lammet fra brystet og ned. Exoskelettet kan registrere når patienten bevæger sig til siden og herved begynder bene at gå, selvom patienten er uden muskelkraft og følesans. Foruden fordele ved at gå, formodes det at, det har en positiv indflydelse på patientens kredsløb, knogler, led og fordøjelse. [6]

%[1]https://books.google.dk/books?id=ha23lFiOGX8C&printsec=frontcover&hl=da&source=gbs_vpt_buy#v=onepage&q&f=false Grundbog om hjælpemidler - til personer med funktionsnedsættelse Åse brandt og lilly Jensen 
%[2] MANGLER EN KILDE!!!!
%[3] MANGLER EN KILDE!!!!
%[4]http://pic.sagepub.com.zorac.aub.aau.dk/content/222/8/1599.full.pdf+htmlA review of exoskeleton-type systems and their key technologies
%[5]http://search.proquest.com.zorac.aub.aau.dk/docview/1647109923?rfr_id=info%3Axri%2Fsid%3Aprimo Robotic exoskeletons: a review of recent progress
%[6]http://www.rm.dk/om-os/aktuelt/nyheder/nyhedsarkiv-2015/december/rygmarvsskadet-lammet-mand-larer-at-ga1/