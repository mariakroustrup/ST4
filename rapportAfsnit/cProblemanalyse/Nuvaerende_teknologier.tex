% !TeX spellcheck = da_DK
\subsection{Nuværende teknologier/hjælpemidler}
% Da der ikke er nogen bevis kur for ALS er teknologierne er pallialiv, dvs. at den er lindrende. 
Som tidligere nævnt er ALS en livstruende sygdom, hvor følgerne sker gradvist, hvilket gør at patienternes funktionelle evner svækkes over sigt, hvorfor der er behov for en række hjælpemidler som helt eller delvist kan være en hjælp i hverdagen. Nogle af hjælpemidlerne er i starten af sygdommen for at patienterne kan klare sig selvstændigt, hvor der senere er behov for andre hjælpemidler samt helt eller delvist hjælp fra en ægtefælde eller plejepersonale. Der anvendes på nuværende tidspunkt teknologiske og personlige hjælpemidler. [2]

\subsubsection{Teknologiske og personlige hjælpemidler}
De mest anvendte hjælpemidler for patienter med ALS er teknologiske som f.eks.  kørestole, toiletstole og stokke. Hjælpemidlerne er alle redskaber som støtter og aflaster patienten. Derudover anvendes der mere personlige hjælpemidler som i stedet er tilpasset patienterne individuelt. Patienterne har på denne måde et særligt behov for hjælpemidler som f.eks. tilpasset kørestole, tilpasset fodtøj og høreapparat. [2]

\subsubsection{Udfordringer og muligheder}
\fxnote{På nuværende tidspunkt har det være svært at finde nogle kilder i forhold til muskelsvind og det at gå, men mine tanker er noget med at koble det sammen ift. det der står under livskvalitet og de nuværende teknologier der er hvor lidt selvstændighed det giver i hverdagen og så senere koble det til at kunne anvendes exoskelet........}
Der er ingen af de nuværende teknologier som giver patienten mulighed for at gå, men hvis patienterne har gangbesvær vil de kunne anvende hjælpemidler som stok og rollator, ellers vil de være nødsaget til at anvende en kørestol for at kunne komme fra A til B. hvilket i forhold til livskvalitet bla..bla..blaa.. er patienterne gerne vil gå.


\subsubsection{Løsning}
Exoskelet som har til opgave at......




Det er her udover påvist at flere af disse metoder ud fra patienternes vurdering medvirker til en forbedret livskvalitet [1]

%[1] http://www.tandfonline.com.zorac.aub.aau.dk/doi/pdf/10.1080/00222895.2014.891970
%[2] https://books.google.dk/books?id=ha23lFiOGX8C&printsec=frontcover&hl=da&source=gbs_vpt_buy#v=onepage&q&f=false Grundbog om hjælpemidler - til personer med funktionsnedsættelse Åse brandt og lilly Jensen 