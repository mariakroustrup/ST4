% mere om, at det kun er lårets muskulatur, der benyttes under squat
Den dynamiske squat-øvelse er en velkendt metode, inden for sport, som kræver et højt niveau af styrke og magt. Squat styrker primært hofter, lår og ryg muskulaturen, som alle er vigtige muskler under løb, spring og løft. Herudover anvendes squat også som et redskab til rehabilitering af knæet, hvilket skyldes den måde knæet udsættes for under en squat \citep{escamilla2001}. 

Knæets funktion for bøjningen af benet ses ved udførelse af en squat-øvelse. En squart udføres ved at stå i en oprejst position
med knæ og hofter fuldt udstrakt. Herefter udføres en
squats i en kontinuerlig bevægelse, indtil den ønskede squat
dybde er opnået, hvorefter der udføres en kontinuerlig bevægelse
tilbage til oprejst position.

Squat kan udføres med varierende grader af fleksion af knæet. De mest anvendt er halv eller fuld squat. Den halve squat indebærer siddende i hug, indtil lårene er parallelle med jorden, hvilket svarer til en fleksion af knæet på omkring $0-100$ $^{\circ}$. Den dybe squat indebærer hug så dyb så muligt, indtil at det posteriore lår og ben kommer i kontakt med hinanden. Den dybe squat anbefales mere trænede personer, hvorfor den halve squat er den typsik foretrukket \citep{escamilla2001}.

Under denne øvelse aktiveres quadricepsmusklerne ved $80-90$ $^{\circ}$ fleksion og er herefter konsistent. Der ses en større aktivering af vastus intermedius, vastus medialis samt vastus lateris, da disse muskler er én ledmuskel, hvor rectus femoris er en to-ledsmuskel. Hamstringmusklerne aktiveres ved en $45$ $^{\circ}$ fleksion \citep{schoenfeld2010}. 
Ved udførelse af en squat-øvelse er det primært lårmusklerne, quadriceps- og hamstringsmusklerne, der aktiveres. 