\section{Amyotrofisk lateral sklerose} \label{sec:ALS}
ALS er en neurodegenerativ sygdom, der påvirker motorneuronerne i hjernen, hjernestammen og rygsøjlen i takt med sygdommens fremskriden, hvilket resulterer i muskelsvaghed \citep{henschke2012}. En illustration af, hvordan ALS påvirker motorneuroner, ses af \autoref{fig:affectedneuron}. De første symptomer på sygdommen er kramper, svaghed samt stive muskler, hvilket kan opstå som muskelsvaghed i arme eller ben, talebesvær eller svaghed i de muskler, som styrer respirationen. Symptomer, der begynder i arme eller ben kaldes ”limb onset ALS”, mens talebesvær samt synkebesvær refereres til ”bulbar onset ALS” \citep{nationalinstitute2016}. 
Symptomerne og følgerne af ALS varierer fra patient til patient, hvorved nogle patienter først oplever muskelsvaghed i deres ben, mens andre oplever muskelsvaghed i deres hænder og arme eller besvær i form af tale- eller synkebesvær \citep{miller2005, nationalinstitute2016}.

\begin{figure}[H]
\centering
\includegraphics[width=1\textwidth]{figures/affectedneuron}
\caption{Tre stadier for en nervecelle samt muskel påvirket af ALS. Det første stadie illustrerer en normal motorneuron samt en upåvirket muskel. Ved andet stadie ses motorneuronet påvirket af ALS, dog ses musklen endvidere upåvirket. I det tredje stadie ses motorneuronet påvirket samt musklen svundet ind. Svindet skyldes en manglende stimulering af musklen som følge af den påvirkede motorneuron \citep{drake2015}.}
\label{fig:affectedneuron}
\end{figure}
 
\noindent
Muskelsvagheden skyldes abnormiteter i de nedre motorneuroner. De nedre motorneuroner er de nerveceller, der videregiver information fra rygmarven til musklerne. Symptomer på abnormiteter i de nedre motorneuroner ses som muskelsvaghed samt muskelkramper og atrofi.
Ligeledes kan de øvre motorneuroner påvirkes. Disse motorneuroner sørger for kommunikationen mellem hjernen og de nedre motorneuroner i rygmarven. Ved abnormitet, opstår komplikationer ved vidersendelse af beskeder til det givne sted. Dette ses som spasticitet samt overdrevne reflekser \citep{nationalinstitute2016}. Opdelingen af de nedre samt øvre motorneuroner ses af \autoref{fig:motorneuroner}.

\begin{figure}[H]
\centering
\includegraphics[width=0.4\textwidth]{figures/motorneuroner.png}
\caption{Illustrerer opdelingen af de nedre samt øvre motorneuroner \citep{miller2005}.}
\label{fig:motorneuroner}
\end{figure}

\noindent
Årsagen til, at ALS opstår er oftest ukendt, dog ses en arvelighed i $5 - 10~\%$ af tilfældene. Herudaf anslås $20~\%$ til at have det muterede Superocide dismutase 1-gen (SOD-1), hvilket resulterer i tab af motorneuroner \citep{miller2005}.

På trods af, at ALS opleves individuelt både i forhold til sygdomsprogressionen samt, hvilke komplikationer de oplever, kan sygdommen inddeles i tre stadier: et tidligt, midter og endeligt stadie. Et flowdiagram af de tre stadier framgår af \autoref{fig:stadier}.

\begin{figure}[H]
\centering
\includegraphics[width=0.8\textwidth]{figures/stadier.png}
\caption{Tre stadier for udviklingen af ALS samt de tilhørende symptomer.}
\label{fig:stadier}
\end{figure}

\noindent
I det tidlige stadie kan patienter ignorere symptomerne, da disse fremstår som milde og kun påvirker mindre dele af kroppen. 
Ved det midterste stadie vil symptomerne begynde at udbrede sig, hvortil nogle muskler paralyseres. Andre muskler vil blive svagere med tiden, hvilket blandt andet kan medføre problemer med synkning og vejrtrækningen. I det endelige stadie vil de fleste voluntære muskler være paralyserede, og det vil derfor forringe deres mulighed for indtage føde eller væske normalt. Herudover vil patienter oftest i dette stadie miste even til selv at trække vejret, og bliver derfor afhængig af ventilationsstøtte \citep{themusculardystrophyassociation2016}.
Den mest almindelige dødsårsag er respirationssvigt, hvilket oftest sker inden for 3 år efter diagnosen er stillet. $25~\%$ af patienterne har en overlevelsesrate på 5 år, og kun $10~\%$ lever længere end 10 år efter diagnosen er stillet \citep{grehl2011, miller2005}.




%Til at starte med kan mindre symptomer som besvær ved at gå op ad trapper opstå. Ligeledes kan patienterne være påvirket af dropfod, når de går. Herefter vil musklerne gradvist blive svagere, og med tiden vil patienterne ikke længere være i stand til at gå.\citep{tidy2015} 