\section{Amyotrofisk lateral sklerose}
%Tilføj billede til afsnittet (billede af motorneuroner - genereret og normalt).
ALS er en neurodegenerativ sygdom, der påvirker motorneuronerne i hjernen og rygsøjlen i takt med sygdommens fremskriden, hvilket resulterer i muskelsvaghed. De første symptomer herpå er kramper, svaghed samt stive muskler, hvilket kan opstå som muskelsvaghed i arme eller ben, talebesvær eller svaghed i de muskler, som styrer respirationen. Symptomer, der begynder i arme eller ben kaldes 'limb onset ALS', mens talebesvær samt synkebesvær refereres til 'bulbar onset ALS'. 
Symptomerne og følgerne af ALS varierer fra patient til patient, hvorved nogle patienter først oplever muskelsvaghed i deres ben, mens andre oplever muskelsvaghed i deres hænder og arme eller besvær i form af tale- eller synkebesvær. \citep{miller2005} \citep{nationalinstitute2016}
 
Muskelsvagheden skyldes abnormiteter i de nedre motorneuroner. De nedre motorneuroner er de nerveceller, der videregiver information fra rygmarven til musklerne. Symptomer på abnormiteter i de nedre motorneuroner ses som muskelsvaghed samt muskelkramper og atrofi.
Ligeledes kan de øvre motorneuroner påvirkes. Disse motorneuroner sørger for kommunikationen mellem hjernen og de nedre motorneuroner i rygmarven. Dette medfører, at beskeden fra hjernen har komplikationer med at komme til det givne sted. Dette ses som spasticitet samt overdrevne reflekser.\citep{nationalinstitute2016}
Årsagen til, at ALS opstår er oftest ukendt, dog ses en arvelighed i 5-10 \% af tilfældene. Herudaf anslås 20 \% til at have det muterede Superocide dismutase 1-gen (SOD-1), hvilket resulterer i tab af motorneuroner. \citep{miller2005}

På trods af, at ALS opleves individuelt både i forhold til sygdomsprogressionen samt, hvilke komplikationer de oplever, vil flere patienter i sidste ende opleve besvær ved at stå og gå samt benytte deres hænder og arme. Herudover opleves synke- og tyggebesvær, hvilket forringer deres evne til at spise normalt. I de seneste stadier af ALS mister patienter even til selv at trække vejret, og bliver derfor afhængig af ventilationsstøtte. Den mest almindelige dødsårsag er respirationssvigt, hvilket oftest sker inden for 3 år efter diagnosen er stillet. 25 \% af patienterne har en overlevelsesrate på 5 år, og kun 10 \% lever længere end 10 år efter diagnosen er stillet. \citep{grehl2011} \citep{miller2005}

