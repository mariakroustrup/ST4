\subsection{Følger}

ALS er en individuel sygdom og sygdommens forløb vil ligeledes variere fra patient til patient. Der kan dog være fællestræk for sygdommens progression for nogle af patienterne. Sygdommen kan inddeles i 3 stadier, et tidligt stadie, et midter stadie og et endeligt stadie. I de tidlige stadier er der mulighed for, at patienterne kan ignorere symptomerne og diagnosticeres oftest først efter dette stadie. [1] Disse symptomer kan være milde og kun påvirke mindre dele af kroppen, hvor musklerne eksempelvis kan være svage eller stive. Dette vil ligeledes have påvirkning på patientens balance. I det midterste stadie vil symptomerne begynde at udbrede sig. Nogle muskler kan være paralyserede, hvor andre eksempelvis kan være upåvirkede. Andre muskler vil blive svagere med tiden og dette vil blandt andet medføre problemer med synkning og vejrtrækningen. I det endelige stadie vil de fleste voluntære muskler være paralyserede og det vil måske ikke være muligt at indtage føde eller væske. Herudover vil det for oftest i dette stadie ikke være muligt at trække vejret grundet respirationssvigt.[1] Andre patienter kan eksempelvis opleve respirationssvigt før tabt muskelfunktion i benene, hvorfor progressionen for sygdommen ikke kan generaliseres for samtlige ALS patienter. Symptomerne behøver nødvendigvis ikke at ramme begge arme eller ben samtidig. Til at starte med kan mindre symptomer som besvær ved at gå op ad trapper opstå. Ligeledes kan patienterne være nødsaget til at trække benet, mens de går. Herefter vil musklerne gradvist blive svagere og med tiden vil patienterne ikke længere være i stand til at gå. [2] 


1: https://www.mda.org/disease/amyotrophic-lateral-sclerosis/signs-and-symptoms/stages-of-als 
2: http://patient.info/health/motor-neurone-disease-leaflet 
