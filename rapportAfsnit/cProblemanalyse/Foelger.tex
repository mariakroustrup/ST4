\subsection{Følger}

ALS er en individuel sygdom og sygdommens forløb vil ligeledes variere fra patient til patient. Dog kan der være fællestræk for sygdommens progression, men med undtagelse af nogle patienter. Man kan inddele sygdommen i 3 stadier, et tidligt stadie, et mellem stadie og et sent stadie. I de tidligste stadier er der mulighed for, at patienterne kan ignorere symptomerne og diagnosticeres oftest efter dette stadie. [1] Disse symptomer kan være milde og kun påvirke mindre dele af kroppen, hvor musklerne eksempelvis kan være svage eller stive. Dette vil ligeledes have påvirkning på patientens balance. I det midterste stadie vil symptomerne begynde at udbrede sig. Nogle muskler kan være paralyserede, hvor andre eksempelvis kan være upåvirkede. Andre muskler vil blive svagere med tiden og dette vil blandt andet medføre problemer med synkning og vejrtrækningen. I de senere stadier vil de fleste voluntære muskler vil være paralyserede og det vil måske ikke være muligt at indtage føde eller væske. Herudover vil det for oftest i dette stadie ikke være muligt at trække vejret grundet respirationssvigt. [1] 

Symptomerne behøver nødvendigvis ikke at ramme begge ben samtidig. Til at starte med at mindre symptomer som besvær ved at gå op ad trapper. Ligeledes kan det være patienterne vil begynde at være nødsaget til at trække benet for at kunne gå. Herefter vil det musklerne gradvist blive svagere og med tiden vil de ikke længere være i stand til at gå. [2]



1: https://www.mda.org/disease/amyotrophic-lateral-sclerosis/signs-and-symptoms/stages-of-als 
2: http://patient.info/health/motor-neurone-disease-leaflet 
