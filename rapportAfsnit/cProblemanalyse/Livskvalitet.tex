\section{Livskvalitet hos ALS patienter}
Livskvaliteten hos patienter med ALS undersøges for at se hvilken påvirkning sygdommen samt dens progression har for patienten. Der er ingen kurerende behandling, dog eksistere der forskellige palliatative behandlinger. Det er bl.a fordelagtigt at kende patienternes livskvalitet for vurdere den optimale palliative behandling \citep{neudert2004}.

Der fremhæves to forskellige typer af livskvalitetsvurderinger: en overordnet liskvalitet og en sundhedshedsrelateret livskvalitet. 
Den 'overordnede' type relatere til patienternes samlede livskvalitet \citep{lse2014, nuebert2004}.   
%Tænker at det ovenstående skal uddybes yderligere ift hvad der definere den overordnede livskvalitet. 

Den sundehedsrelaterede livskvalitet relatere til de fysiologiske og mentale aspekter ved sygdommmen. Da ALS påvirker patients fysiske kontrol, ses der et fald af denne type livskvalit, i takt med sygdommens frembrud. \citep{lse2014} 


Tiltrods for faldet af den sundhedsrelateret livskvalitet, forbliver den overordnet livskvalitet stabil \citep{lse2014, nuebert2004}. Årsagen hertil er bl.a. at den sociale støtte blever mere omfattende i tagt med progressionen, og dermed kompencere for de manglende fysiske egenskaber. \citep{lse2014}   

% Det er min hensigt at lave en form for afgrænsning til de fysiske mangler ved at tage udgangspunkt i den sundhedsrelateret livskvalitet og det med at den forringes i takt med at de oplever større svaghed i musklerne. 
%Tænker det er en afgrænsning vi bliver nød til at fortages os, for jeg synes ikke at kunne finde litteratur der giver den nødvendige relation mellem det at jeg skal fokusere på besværligheden i at gå.