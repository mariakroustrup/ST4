\subsection{Livskvalitet hos ALS-patienter}
% mere konkret omkring deres livskvalitet og familieforhold
% tilføj tabel 2 fra ilse2015
Livskvaliteten hos patienter med ALS undersøges for at vurdere, hvilken påvirkning sygdommen samt dens progression har på patienten. Der er ingen behandling for at stoppe sygdomsprogressionen, men der eksisterer forskellige palliatative behandlinger \citep{neudert2004}. Det er fordelagtigt at kende patienternes livskvalitet for at vurdere den optimale palliative behandling \citep{ilse2015}.

Livskvalitet defineres ud fra en persons fysiske sundhed, psykologiske tilstand, grad af selvstændighed, sociale relationer og personlig tro \citep{pagnini2013}.
\noindent
Der kan fremhæves to forskellige typer af livskvalitetsvurderinger: en overordnet livskvalitet og en sundhedshedsrelateret livskvalitet. 
Den overordnede livskvalitet relaterer til patienternes samlede livskvalitet, og den sundhedsrelaterede livskvalitet dækker over de fysiologiske og mentale aspekter ved sygdommen \citep{ilse2015, neudert2004}. Da ALS påvirker patienters fysiske formåen, ses der et fald i denne type livskvalitet, som sygdommen fremskrider \citep{ilse2015}. Dette fremgår ligeledes af \autoref{tab:livskvalitet}, der viser en forringet livskvalitet hos ALS-patienter når der sammenlignes med resten af befolkningen. 

\begin{table}[H]
\centering
\begin{tabular}{l|>{\centering\arraybackslash} p{4,5cm}|>{\centering\arraybackslash} p{4,5cm}}
 
                                                                         & ALS-patienter                                    & Normativ tysk population                                   \\
                                                                         \hline
                                                                  
Mobilitet                                                                & 83,7 \%                                          & 16,6 \%                                                    \\
Selvpleje                                                                & 77,6 \%                                         & 2,9 \%                                                     \\
Normale aktiviteter                                                      & 85,7 \%                                          & 10,2 \%                                                    \\
Smerte eller ubehag                                                      & 61,2 \%                                          & 27,9 \%                                                    \\
Angst eller depression                                                   & 67,4 \%                                          & 4,4 \%                                                    \\
\end{tabular}
\caption{Moderate eller alvorlige problemer målt ud fra europæisk livskvalitetsvurdering. Tabellen sammenligner livskvaliteten for ALS-patienter med livskvaliteten for den tyske population. Det ses heraf at ALS-patienter har en forringet livskvalitet i forhold til den resterende tyske befolkning \citep{ilse2015}. (Revideret)}
\label{tab:livskvalitet}
\end{table}

\noindent
Livskvaliteten vurderes ud fra mobilitet, selvpleje, udførelse af normale aktiviteter, oplevelse af smerte eller ubehag samt diagnoser som angst og depression, hvor næsten tre gange så mange ALS-patienter lever med disse problemer sammenlignet med den resterende befolkning.

Til trods for, at der sker et fald i den sundhedsrelaterede livskvalitet, er der tidligere vist, at den overordnede livskvalitet forbliver stabil \citep{ilse2015, neudert2004}. Dette kan forklares ved et ”response shift” eller ”frame shift”, der er en måde at håndtere sin sygdom, hvor social støtte under sygdomsforløbet vægtes højere end normalt i bestemmelsen af livskvalitet \citep{ilse2015}. Af denne grund foreslås det, at faldet i sundhedsrelateret livskvalitet i forhold til mobilitet og selvhjælp afhjælpes ved teknologiske hjælpemidler. På denne måde vil ALS-patienternes sociale interaktioner kunne have fokus på deres sociale netværk, da disse sociale interaktioner er begrænsede på baggrund af ALS \citep{ilse2015,tramonti2012}.







% Jeg ved ikke, om dette er nok. Men jeg tænker, at vinklingen er mere som den, vi snakkede om. Der kan sandsynligvis sagtens tilføjes noget hertil - evt. tænker jeg, om man vil kunne tilføje (noget af?) tabel 2 fra ilse2015 for at få nogle tal på bordet. 
% Mads: Det er min hensigt at lave en form for afgrænsning til de fysiske mangler ved at tage udgangspunkt i den sundhedsrelateret livskvalitet og det med at den forringes i takt med at de oplever større svaghed i musklerne. 
%Tænker det er en afgrænsning vi bliver nød til at fortages os, for jeg synes ikke at kunne finde litteratur der giver den nødvendige relation mellem det at jeg skal fokusere på besværligheden i at gå.