\subsection{Test af accelerometer}
I dette projekt anvendes to accelerometer som sensorer til opsamling af signaler. For at kunne anvende et accelerometer er det vigtigt at kende forskellige tolerancer i forhold til deres datablade, hvorfor et forsøg udføres for at kunne tage højde for disse under videre forsøg.

\subsubsection{Formål}
Denne test har til formål at identificere spændingen i henholdsvis $0^{\circ}$ og $90^{\circ}$, hvorudfra vinkler kan beregnes. Derudover identificeres støjsignaler i outputsignalet, bestemmelse af den maksimale og minimale output samt offsettet og sensitivitet, så dette kan tages højde ved i pilotforsøget \autoref{sec:pilotforsøg} og senere forsøg. 

\begin{enumerate}
\item Identificering af spænding ved $0^{\circ}$ og $90^{\circ}$
\item Identificering af støj i outputsignaler for accelerometeret
\item Identificering af maksimale og minimale output ved rotation
\item Identificering af offset og sensitivitet for accelerometeret i forhold til databladet.
\end{enumerate}

\subsubsection{Materialer}
\begin{itemize}
\item Accelerometre ADXL$335$
\item Tape
\item Computer med MATLAB
\item CY$8$CKIT-$042$-BLE
\item Vinkel
\item Vaterpas
\item Breadboard
\item $0.1$ \mikroF kondensator \fxnote{Skal vi anvende dette?}
\item $\pm 5.5 V$ spændingsforsyning
\end{itemize}

\subsubsection{Metode}
Alle målingerne foretages i en akse på accelerometeret, der er på baggrund af databladet over accelerometeret valgt at anvende y-aksen på det aksialeaccelerometer. 
\begin{enumerate}
\item Der foretages målinger ved $0$ g-påvirkning, hvilket svarer til en hældning på XX og ved $1$ g-påvirkning, hvilket svarer til en hældning på XX. 
\item Støjsignaler i outputtet identificeres ved måling af baseline uden nogen g-påvirkningen, hvorefter det måles ved $1$ g-påvirkningen. 
\item Den maksimale og minimale output måles ved at rotere accelerometeret fra $0^{\circ}$ og $90^{\circ}$.
\item På baggrund af de forrige mål kan offsettet og sensitiviteten udregnes.
\end{enumerate}

\subsubsection{Forsøgsopstilling}
Forsøgsopstillingen er den samme for alle formål og udføres 2 gange, da der anvendes to accelerometre.
\begin{itemize}
\item Accelerometeret sættes på vinklen med tape, så y-aksen vender opad. Herefter sættes det i vater ved brug af vaterpas
\item Accelerometeret tilkobles breadboard
\item Accelerometeret forbindes til computer
\item ...\fxnote{der mangler lidt information}
\end{itemize}

\subsubsection{Fremgangsmåde}
Der foretages  3 forskellige forsøg.
\begin{itemize}
\item Identificering af spænding ved $0^{\circ}$ og $90^{\circ}$ og støj signaler
\begin{itemize}
\item Baseline i 10 sekunder ved $0$ g-påvirkning svarende til en vinkel på $0^{\circ}$, hvorved accelerometeret placeres fladt på bordet
\item Baseline i 10 sekunder ved $1$ g-påvirkning svarende til en vinkel på $90^{\circ}$, hvorved accelerometeret holdes oprejst.
\end{itemize}
\item Identificering af maksimale og minimale output 
\begin{itemize}
\item Rotation i 10 sekunder fra $0^{\circ}$ til $90^{\circ}$ til højre
\end{itemize}
\end{itemize}


\subsection{Behandling af data}
