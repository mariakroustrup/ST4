\section{Pilotforsøg}
%Nirushas udgave
I dette projekt udføres et pilotforsøg for identificering af støj samt andre uønskede signaler ved anvendelse af sensorer. Pilotforsøget danner grundlag for optimering af kravspecifikationerne i de forskellige blokke. Herudover undersøges elektrodernes placering for at opnå det bedst mulige signal under udførelse af en squat-øvelse.
%Signes udgave
I dette projekt udføres et pilotforsøg for identificering af støj samt andre uønskede signaler ved anvendelse af sensorer. Pilotforsøget danner grundlag for optimering af kravspecifikationerne i de enkelte blokke. Derudover undersøges det, hvor elektroderne skal placeres for at opnå det bedst mulige signal under udførelse af en squat-øvelse.


\subsection{Formål}
Der anvendes en EMG-forstærker og et accelerometer som sensorer. På baggrund af dette opstilles følgende formål for de enkelte sensorer.  

\subsubsection{EMG-forstærker}
\begin{enumerate}
\item Opsamling af signal fra rectus femoris og biceps femoris
\begin{itemize}
%Nirushas udgave
\item Identificering af elektrodernes placering
\item Sammenligning af muskelaktivitet under en squat-øvelse 
%Signes udgave
\item Identificere placeringen af elektroder
\item Sammenligne muskelaktivitet oprejst og i en squat-øvelse 

\end{itemize}
\item Identificering af støjsignaler
\end{enumerate}


\subsubsection{Accelerometer}
\begin{enumerate}
%Nirushas udgave
\item Identificering af position under en squat-øvelse
\item Identificering af støjsignaler
%Signes udgave
\item Identificere position af knæleddet siddende i en squat-øvelse
\item Identificere støj ved opsamling af signaler

\end{enumerate}


\subsection{Materialer} 
\begin{itemize}
\item EMG-forstærker
\item Elektroder
\item Desinfektionsservietter
\item Skraber
\item Tusch 

\item Accelerometer ADXL335Z
\item Tape
\item Ledninger

\item Computer
\item CY8CKIT-042-BLE
\end{itemize}

\subsection{Metode}

%Nirushas udgave
For at identificere den bedste mulige elektrodeplacering optages EMG signaler fra forskellige placeringer på de to muskler. 
For at simulere den påvirkning som accelerometeret udsættes for og derved identificere det maksimale og minimale outputsignal roteres accelerometeret i en langsom rotation fra 0 $^{\circ}$ til 90 $^{\circ}$ til både højre og venstre. Herudover måles accelerometerets påvirkning i henholdsvis 0 og 1 g-påvirkning, for at identificere accelerometeres påvirkning og hvorledes dette stemmer overens med databladet. 
For at identificere støj fra EMG forstærkeren optages aktivitet i musklerne under en squat-øvelse.
 
%Signes udgave
For at identificere den bedste placering af elektroder optages EMG-signaler fra forskellige placeringer på de to muskler. 
For at simulere den påvirkning som accelerometeret udsættes for og derved identificere det maksimale og minimale outputsignal roteres accelerometeret i en langsom rotation fra 0 $^{\circ}$ til 90 $^{\circ}$  til både højre og venstre. Herudover måles accelerometeret påvirkning i henholdsvis 0 og 1 g-påvirkning for at identificere accelerometeres påvirkning og hvorledes dette stemmer overens med databladet. 
For at identificere støj fra EMG-forstærkeren optages aktivitet i musklerne i en squat-øvelse.


\subsection{Forsøgsopstilling}
Forsøgsopstilling er for den primære udførelse af forsøget. Nogle af processerne gentages for at kunne sammenligne de forskellige målinger, og derved få et bedre resultat.

%Nirushas udgave
\subsubsection{EMG-forstæker}
Rectus femoris og biceps femoris identificeres, den ønskede placering af elektroderne markeres med tusch. Herefter fjernes eventuelle hår og døde hudceller ved brug af skraber. Huden desinficeres herefter ved brug af desinficeringsservietter og elektroderne påsættes. Den røde ledning påsættes rectus femoris/bicep femoris og den grønne ledning påsættes rectus femoris/bicep femoris. Den sorte ledning påsættes (tibia) og anvendes som referencepunkt.
%Signes udgave
\subsubsection{EMG-forstæker}
Rectus femoris og biceps femoris identificeres, den ønskede placering af elektroderne markeres med tusch. Herefter fjernes eventuelle hår og døde hudceller ved brug af skraber. Huden desinficeres herefter ved brug af desinficeringsservietter og elektroderne påsættes. Den røde ledning påsættes rectus femoris/bicep femoris og den grønne ledning påsættes rectus femoris/bicep femoris. Den sorte ledning påsættes patella og anvendes som referencepunkt.


\subsubsection{Accelerometer}
Accelerometeret påsættes siden af låret, så accelerometer måles i xyz-plan, hvorved der måles i den vertikale retning. Der sørges for,  at accelerometeret befinder sig i 0 g påvirkning ved starten af forsøgets udførelse, hvorved accelerometeret er kaliberet. 

\subsubsection{Opstilling}
\begin{itemize}
\item Identificering af musklerne rectus femoris og biceps femoris 
\item Elektrodernes placering markeres
\item Huden skrabes og desinficeres
\item Elektroderne påsættes
\item Ledningerne påsættes elektroderne
\begin{itemize}
\item Den røde/grønne ledning på rectus femoris
\item Den røde/grønne ledning på biceps femoris
%Nirushas udgave
\item Den sorte ledning/reference på tibia??
%Signes udgave
\item Den sorte ledning/reference på patella \fxnote{positiv/negativ/ground}

\end{itemize} 
\item Accelerometeret på sættes patella ved en 0 g påvirkning i x,y,z retning
\end{itemize}


\subsection{Fremgangsmåde}
Fremgangsmåden udføres XX antal gange, hvorved der på baggrund af målingerne foretages en gennemsnitsværdiberegning.
%Nirushas udgave
\subsubsection{EMG/EMG forstærker}
EMG måling: Squat-øvelsen starter stående, hvorefter forsøgspersonen langsomt udføre bevægelsen og dermed kommer ned i en dybere squat, hvor der undervejs foretages målinger.
%Signes udgave
\subsubsection{EMG/EMG-forstærker}
EMG måling: 10-sekunders målinger trinvist under udførelse af en squat-øvelse. 


\subsubsection{Accelerometer}
Påvirkning i 0 og 90 $^{\circ}$.
Påvirkning af rotation fra 0 til 90 $^{\circ}$ til både højre og venstre.
Optag 30 sekunder ved 0 og 90 $^{\circ}$ .
Optag rotation: baseline 10 sekunder, rotation 10 sekunder, baseline 10 sekunder




