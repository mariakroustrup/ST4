% !TeX spellcheck = da_DK
\subsection{Indledning}
%Hvad er muskelsvind (overordnet)
%Prævalens/incidens i Danmark?
%Hvorfor er det et problem samfundsmæssigt?

%Dette afsnit skal starte mere overordnet, vil gerne have byttet om så der først forklares hvor mange der lider af problemet, og her efter forklaring af hvad muskelsvind er mere generelt....Og hvad er autoimmune sygdomme, eventuelt en fxnote. Sammenligning med kørestol og levetid hænger måske et særligt godt sammen. Vær opmærksom på talesprog brug af altså.  

Muskelsvind dækker over omkring 25 forskellige neuromuskulære sygdomme, altså sygdomme der påvirker samspillet med nerver og muskler. De fleste af disse sygdomme opstår som følge af gendefekter, og andre skyldes autoimmune sygdomme. Omkring 3000 mennesker i Danmark har en muskelsvindssygdom og over 80 \% af disse har behov for hjælpemidler og behandling, ofte i form af kørestole, respirationshjælp og fysioterapi. Dette gør, at flere med muskelsvind kan leve lige så længe, som andre kan, selvom sygdommen ikke kan helbredes. \citep{hvadermuskelsvind2016,sygdomsbeskrivelser2016}

%Hvilke forskellige former for muskelsvind findes? 
%Afgrænsning til ALS
Der er stor forskel på de enkelte muskelsvindssygdomme, og derfor også på hvordan de udvikler sig og deres konsekvenser. De fleste muskelsvindssygdomme er ikke livstruende, når de rette hjælpemidler benyttes, men en af de mest alvorlige muskelsvindssygdomme, der medfører dødsfald grundet sygdommens komplikationer, er Amyotrofisk Lateral Sklerose (ALS).  \citep{hvadermuskelsvind2016}

%Kort om ALS: prævalens/incedens, hvorfor dette specielt er et problem
% Hedder det progredierende - find ud af det? 
% Destrueres -  degenerere.
ALS er en hurtigt progredierende neurodegenerativ sygdom, der nedbryder motorneuroner i hjernen, hjernestammen og rygmarven, som kontrollerer kroppens muskler. Det betyder, at nerveceller destrueres, så der opstår atrofi. I Danmark er incidensen af ALS 1-3 per 100.000, og prævalensen er 3-7 per 100.000. ALS har en dårlig prognose, og den gennemsnitlige levetid for patienter er tre til fem år efter symptomdebut, hvorefter motoriske neuroner er nedbrudt i en sådan grad, at der ofte opstår terminalt respirationssvigt. \citep{russell2015, morris2015}

ALS-patienter er udfordrede under progressionen af sygdommen, hvor musklerne degenererer, da sygdommen udvikler sig hurtigt. Af denne grund mister patienterne gradvist kontrol over sine muskler og almindelige kropsfunktioner. Disse funktioner ønskes opretholdt, hvilket leder frem til følgende initierende problemstilling:

??? 
