% !TeX spellcheck = da_DK
\subsection{Indledning}
%Hvad er muskelsvind (overordnet)
%Prævalens/incidens i Danmark?
%Hvorfor er det et problem samfundsmæssigt?

%Dette afsnit skal starte mere overordnet, vil gerne have byttet om så der først forklares hvor mange der lider af problemet, og her efter forklaring af hvad muskelsvind er mere generelt....(Hvorfor? Valgte at lægge fokus anderledes, fordi der er så få, der lider af det?) Og hvad er autoimmune sygdomme, eventuelt en fxnote.  Sammenligning med kørestol og levetid hænger måske et særligt godt sammen. (Har indskrevet "hjælpemidler", hjælper det?) Vær opmærksom på talesprog brug af altså.  

<<<<<<< HEAD
I Danmark er omkring 3.000 mennesker diagnosticeret med en muskelsvindssygdom. Muskelsvind dækker over omkring 25 forskellige neuromuskulære sygdomme, hvilket er sygdomme, der påvirker samspillet mellem nerver og muskler. De fleste af disse sygdomme opstår som følge af gendefekter og andre skyldes autoimmune sygdomme, hvor immunsystemet reagerer på kroppens eget væv. Over 80 \% af patienterne med muskelsvind har behov for hjælpemidler og behandling, ofte i form af kørestole, respirationshjælp og fysioterapi. Disse hjælpemidlerne og behandlingsformer gør, at flere med muskelsvind kan leve lige så længe, som andre kan, selvom sygdommen ikke kan helbredes. \citep{hvadermuskelsvind2016,sygdomsbeskrivelser2016}
=======
I Danmark lever omkring 3.000 mennesker med en muskelsvindssygdom, hvilket dækker over omkring 25 forskellige neuromuskulære sygdomme. Dette er sygdomme, der påvirker samspillet mellem nerver og muskler. De fleste neuromuskulære sygdomme opstår som følge af gendefekter, og andre skyldes autoimmune sygdomme, hvor immunsystemet reagerer på kroppens eget væv. Over 80 \% af patienter med muskelsvind har behov for hjælpemidler og behandling, ofte i form af kørestole, respirationshjælp og fysioterapi. Hjælpemidlerne og behandlingerne gør, at flere med muskelsvind kan leve lige så længe, som andre kan, selvom sygdommen ikke kan helbredes. \citep{hvadermuskelsvind2016,sygdomsbeskrivelser2016}
>>>>>>> origin/master

%Hvilke forskellige former for muskelsvind findes? 
%Afgrænsning til ALS
De enkelte muskelsvindssygdomme er forskellige, og der er derfor forskel på udviklingen og konsekvenserne af dem. De fleste muskelsvindssygdomme er ikke livstruende, hvis de rette hjælpemidler benyttes, men en af de mest alvorlige muskelsvindssygdomme, der medfører dødsfald grundet sygdommens komplikationer, er Amyotrofisk Lateral Sklerose (ALS).  \citep{hvadermuskelsvind2016}

%Kort om ALS: prævalens/incedens, hvorfor dette specielt er et problem
% Hedder det progredierende - find ud af det? Ja, det gør det. 
% Destrueres -  degenerere.
ALS er en hurtigt progredierende neurodegenerativ sygdom, der nedbryder motorneuroner i hjernen, hjernestammen og rygmarven, som kontrollerer kroppens muskler. Det betyder, at nerveceller degenereres, så der opstår atrofi. I Danmark er incidensen af ALS 1-3 per 100.000 (indbyggere?), og prævalensen er 3-7 per 100.000 (indbyggere?). ALS har en dårlig prognose, og den gennemsnitlige levetid for patienter er tre til fem år efter symptomdebut, hvorefter motoriske neuroner er nedbrudt i en sådan grad, at der ofte opstår terminalt respirationssvigt. \citep{russell2015, morris2015}

ALS-patienter er udfordrede under progressionen af sygdommen, hvor musklerne degenererer, da sygdommen udvikler sig hurtigt. Af denne grund mister patienterne gradvist kontrol over sine muskler og almindelige kropsfunktioner. Disse funktioner ønskes opretholdt, hvilket leder frem til følgende initierende problemstilling:

Hvilken indvirkning har amytrofisk lateral sklerose (på patienterne?), og hvilke muligheder er der for opretholdelse af funktioner, der er tabt grundet mistet muskelkraft. 
 
