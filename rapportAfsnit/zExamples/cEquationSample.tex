\section{Equation Sample}

\noindent
Some explanation:
%
\begin{align}
\unit{Unit}
[Equation]&=[Number]
\label{eq1}
\end{align}

\noindent
Some other explanation:
%
\begin{align}
\unit{Unit}
[Equation]&=[Number]
\label{eq2}
\end{align}

\noindent
Yet an explanation:
%
\begin{align}
\unit{Unit}
\text{You see? } [Equation]&=[Number]
\label{eq3}\\
%
\unit{Unit}
\text{Unit isn't aligned } \textbf{:( } \: [Equation]&=[Number]
\label{eq4}	   %
\end{align}	 	%
			  	 %
\noindent	   	  %
Explanation!:	   %
%				 	%
\begin{align}	  	 %
\unit{Unit}		   	  %
[Equation]&=[Number]   %	%-----------------------NOTES-----------------------%
\label{eq5}\\		 	%	% The &-sign aligns the equal signs.				%
%					  	 %	%													%
\unit{Unit}			   	  % % \unit{} will not be alligned by default,			%
[Equation]&=[Number]	   %% however the macro can be modded as described 		%
\label{eq6}\\			    % in macros.tex. This will allign the units,		%
%						    % but generate errors.								%
\unit{Unit}				    %													%
[Equation]&=[Number]		% \noindent should generally be used befor			%
\label{eq7}					% oneliners after equations, figures and tables.	%
\end{align}					%---------------------------------------------------%
\\
%
\noindent
\eqref{eq1}\\
\noindent
\eqrefTwo{eq1}{eq2}\\
\noindent
\eqrefThree{eq1}{eq2}{eq3}\\
\noindent
\eqrefFour{eq1}{eq2}{eq3}{eq4}\\
\noindent
\eqrefFive{eq1}{eq2}{eq3}{eq4}{eq5}\\
\noindent
\eqrefSix{eq1}{eq2}{eq3}{eq4}{eq5}{eq6}\\
\noindent
\eqrefSeven{eq1}{eq2}{eq3}{eq4}{eq5}{eq6}{eq7}\\
\noindent
\Eqref{eq1}\\
\noindent
\EqrefTwo{eq1}{eq2}\\
\noindent
\EqrefThree{eq1}{eq2}{eq3}\\
\noindent
\EqrefFour{eq1}{eq2}{eq3}{eq4}\\
\noindent
\EqrefFive{eq1}{eq2}{eq3}{eq4}{eq5}\\
\noindent
\EqrefSix{eq1}{eq2}{eq3}{eq4}{eq5}{eq6}\\
\noindent
\EqrefSeven{eq1}{eq2}{eq3}{eq4}{eq5}{eq6}{eq7}
\pagebreak