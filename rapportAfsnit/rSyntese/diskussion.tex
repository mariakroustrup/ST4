\section{Diskussion}
Formålet med dette projekt er at udvikle et system, som kan opsamle signaler fra kroppen, hvor der efterfølgende foretages digital signalbehandling med henblik på datakommunikation. På baggrund af dette er der udviklet et systemet til at opsamle signaler fra rectus femoris, samt signaler fra accelerometrer til at informere om henholdsvis muskelaktivitet og vinkelberegning. Ud fra digital signalbehandling af disse signaler er der udviklet et system, som har til formål at støtte knæets omkringliggende muskler hos ALS-patienters i forbindelse med udførelsen af en squat-øvelse. På baggrund af teori, implementering og test af systemets enkelte blokke fremgår det, at kravene til de enkelte blokke er overholdt. Der er dog nogle områder, hvor andre alternativer kan overvejes for mulige forbedringer af hele systemet. 

\subsection{Implementering}
Systemet er udviklet, så størstedelen af signalbehandlingen foregår digitalt, da fokus i forhold til studieordningen er på dette område. Hvis dette ikke havde været et krav for dette semester, kunne det have været en fordel, at nogle af blokkene var designet analogt, hvorved det kunne være nemmere at finde samt rette op på eventuelle fejl i systemet. Et eksempel herpå er i forhold til opsamling af signaler fra EMG og accelerometrer. 

Den udleverede EMG-forstærker ensretter, forstærker og envelopefiltrerer signalet, hvorved tolerancen oplyst i databladet er accepteret. Ved at holde de enkelte blokke hver for sig ved at implementere det analogt, vil det være muligt at måle de enkelte blokke og derved justere og opstille krav indtil den ønskede effekt af blokken er opfyldt. Det er dog vurderet, at eventuelle forsinkelser og ændringer ikke vil have den store betydning ved at anvende EMG-forstærkeren, da signalet er meget lavfrekvent og det vil derfor ikke have en betydning for repræsentation af muskelaktiviteten. 

For at give en bedre opløsning af accelerometeret kunne der være implementeret et gain inden opsamlingen, hvorved der opnås en bedre repræsentation af accelerometre-signalerne. Dette blev ikke implementeret, da dette yderligere ville kræve en offsetjustering af signalet. Offsettet er teoretisk ved en påvirkning af $0~g$ den halve spændingsforsyning. Uden denne offsetjustering vil det kunne resultere i, at signalet vil gå i mætningen, da dette vil kunne overstige ADC'ens arbejdsområde alt efter hvor meget signalet forstærkes, hvorved der yderligere skal tages højde for denne parameter.

Der kunne i stedet for accelerometrene være implementeret andre sensorer som gyroskop og goniometer. Ved et gyroskop anvendes impulsmomentbevarelse, hvorved der kan udregnes hvor meget patienten har bevæget sig under en squat-øvelse. Ved goniometer er det muligt at se, hvilken vinkel knæet har i oprejst position og hvor meget det så har bevæget sig. Da der ønskes at beregne vinklen gennem hele øvelsen, blev disse sensorer fravalgt. Udover disse kriterier  var accelerometrene også til rådighed. Da det er muligt at videreudvikle systemet så det kan benyttes under gang, vil det også være fordelagtigt at anvende accelerometre, det skal dog vurderes om der skal anvendes andre, da det kræver at accelerometrene har et større arbejdsområde end $\pm3~g$.

\subsection{Datakommunikation}
I projektet er der valgt at anvende trådløs kommunikation, da dette var en mulighed ved den valgte løsningsstrategi. Dette er nødvendigvis ikke den mest hensigtsmæssige metode, da der kan opnås respons tid og forsinkelse, hvilket kan få betydning for den videre kommunikation og udførelse. En løsning på dette vil være ikke at koble de enkelte komponenter til det samme baseboard og koble dem mellem hinanden, hvorved der ikke opnås en trådløs forbindelse. Denne løsningsmetode vil kunne give nogle begrænsninger i forhold til rækkevidde, dette vil dog også kunne ske ved den trådløse kommunikation, hvorfor dette ikke vurderes at være et større problem i forhold til implementering. 


\subsection{Batteridrevet}
På nuværende tidspunkt er systemet batteridrevet, hvilket giver noget med at undersøge batteriets leve tid samt hvordan systemet kan bliver mere effektiv i forhold til ikke at bruge at batteri???

\subsection{Brugervenlighed}
Systemet vil ikke have nogen gavnlig effekt på ALS-patienter på nuværende tidspunkt, da systemet ikke er færdig udviklet. En prototype af systemet, herunder ved et exoskelet vil være med til at støtte musklerne omkring knæet  under udførelse af squat-øvelse, men vil ikke kunne have nogen gavnlig effekt i forhold til ALS-patienter, hvorfor det ikke er særligt brugervenligt for patienterne. Da det har vist sig at være muligt at udvikle et system, som kan aflaste knæets omkringliggende muskler vurderes det at være muligt at udvikle et system som kan støtte musklerne under gang, hvorved systemet vil være mere essentielt at anvende for ALS-patienter.  
For at kunne gøre dette skal der tages højde for flere ukendte parametre, som kan variere fra patient til patient. 


\subsection{Samlet systemtest}