\section{Perspektivering}
I dette afsnit vil projektet perspektiveres for at reflektere over de forskellige aspekter, der bør undersøges for at skabe et færdigudviklet produkt, der kan anvendes af ALS-patienter. 

Systemet er udviklet til at kunne hjælpe ALS-patienter ved at aflaste deres lårmuskulatur under en squat-øvelse. Der er ikke udviklet en prototype, der skal kunne muliggøre dette, hvorfor systemet skal videreudvikles, således det er anvendeligt uden at være til gene for brugeren. Systemet optimeres og forbedres på flere områder, da det har nogle begrænsninger, der betyder, at systemet ikke kan anvendes ved en vinkel over $180^{\circ}$ og under $90^{\circ}$. Dette skaber derfor nogle begrænsninger for brugerens bevægelig. Herudover vil det være fordelagtigt at sende en advarsel inden grænserne på $180^{\circ}$ og  $90^{\circ}$ er overskrevet i form af en vibration. Brugeren skal på nuværende tidspunkt selv starte og stoppe systemet, hvilket kan videreudvikles til en automatisk funktion, når brugeren udfører en bevægelse.  


Ydermere vil systemet kunne videreudvikles, således det vil være muligt at anvende under gang for ALS-patienter, og derved støtte deres muskulatur, da det på nuværende tidspunkt kun er muligt at udføre en squat-øvelse. En implementering i forhold til sikkerhed kan være en udvikling, hvor en alarm starter i tilfælde af, at brugeren mister balancen eller falder under gang. 




