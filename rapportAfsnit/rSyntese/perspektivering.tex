\section{Perspektivering}
I dette afsnit vil projektet perspektiveres for at reflektere over de forskellige aspekter, der bør undersøges for at skabe et færdigudviklet produkt, der kan anvendes af ALS-patienter. 

Systemet er udviklet til at kunne hjælpe ALS-patienter ved at aflaste deres lårmuskulatur under en squat-øvelse. Der er ikke udviklet en prototype, der skal kunne muliggøre dette, hvorfor systemet skal videreudvikles, således det er anvendeligt uden at være til gene for brugeren. Ydermere vil systemet kunne videreudvikles, således det vil være muligt at anvende under gang for ALS-patienter, og derved støtte deres muskulatur, da det på nuværende tidspunkt kun er muligt at udføre en squat-øvelse. Systemet kan optimeres og forbedres på flere områder, da det har nogle begrænsninger, der betyder, at systemet ikke kan anvendes ved en vinkel under $90^{\circ}$ og over $180^{\circ}$. Dette skaber derfor nogle begrænsninger for brugerens bevægelig. Herudover vil det være fordelagtigt at sende en advarsel inden grænserne på $90^{\circ}$ og $180^{\circ}$ er overskrevet i form af vibration eller lyd. Brugeren skal på nuværende tidspunkt selv starte og stoppe systemet, hvilket kan videreudvikles til en automatisk funktion, når brugeren udfører en bevægelse. Hvis systemet skal være mere effektivt, kan det i stedet udvikles til, at brugeren eksempelvis skal være i stand til at tænde og slukke systemet trådløst ved andvendelse af en mobiltelefon. 

En implementering i forhold til sikkerhed kan være en udvikling, hvor en alarm starter i tilfælde af, at brugeren mister balancen eller falder under gang. Dette vil kræve, at systemet ikke har nogle begrænsninger i forhold til vinklen. 

Med mere tid og flere ressourcer kunne de forskellige musklers sammenhæng i låret været undersøgt, således det ville have været muligt at bevæge sig under $90^{\circ}$ og over $180^{\circ}$. Herudover vil det kunne undersøges, hvordan miktrokontrolleren kan  (Der skal skrives mere..) 




