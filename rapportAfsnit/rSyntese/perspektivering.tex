\section{Perspektivering}
I dette afsnit vil projektet perspektiveres for at reflektere over de forskellige aspekter, der burde undersøges for at skabe et færdigudviklet produkt, der kan anvendes af ALS-patienter. 

Systemet er udviklet til at kunne hjælpe ALS-patienter ved at aflaste deres lårmuskulatur under en squat-øvelse. Der er ikke udviklet en prototype, der skal kunne muliggøre dette. Herudoer kan systemet optimeres og forbedres på flere områder. Systemet har nogle begrænsninger, der betyder, at brugeren ikke kan bevæge sig over $180^{\circ}$ og under $90^{\circ}$. Dette skaber derfor nogle begrænsninger for brugeren. Det vil være en mulighed at sende en advarsel inden grænserne er overtrådt. Brugeren skal herudover selv starte og stoppe systemet, hvilket kan videreudvikles til en automatisk funktion, når brugeren udfører en bevægelse. 


Systemet er i forvejen begrænsende, da da kun er muligt at udføre en squat-øvelse på nuværende tidspunkt. Her kan systemet videreudvikles, således det vil være muligt at anvende under gang for ALS-patienter, og derved støtte deres muskulatur. 

