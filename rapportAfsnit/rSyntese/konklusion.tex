\section{Konklusion}
I dette afsnit konkluderes på det udarbejdede system og problemformuleringen, som begge diskuteres i \autoref{sec:diskussion}. 

I projektet er der udviklet et system, der kan optage EMG-signaler fra rectus femoris samt signaler fra accelerometre. EMG-signaler viser, hvorvidt muskelaktiviteten er stigende eller faldende samtidigt med, at signaler fra accerelerometrene anvendes til beregning af knæets vinkel under en squat-øvelse. Systemet er udviklet med henblik på at kunne hjælpe ALS-patienter til aflastning af muskulaturen omkring knæet. Dermed udarbejdes systemet ud fra projektets problemformulering. 

Systemets blokke, der består af signalopsamling, spændingsforsyning, ADC, digital filtrering, accelerometeralgoritme, EMG-algoritme og trådløs kommunikation er designet, implementeret samt testet. Disse blokke evalueres ud fra testene for at vurdere, hvorvidt de opstillede kravspecifikationer for blokkene opfyldes. Ud fra dette ses det, at ét af kravene til trådsløs kommunikation ikke opfyldes, da det ikke vælges at benytte BLE-donglen.
Hele systemets forsinkelse, uden trådløs kommuniaktion, blev målt til $832~\mu s$. Derfor er kravet om en maksimal forsinkelse på $100~ms$ opfyldt.

Det kan ud fra evalueringen af testene konkluderes, at de resterende kravspecifikationer for de enkelte blokke er opfyldt. 

Det samlede system er herefter testet med et kendt inputsignal samt et inputsignal fra en bruger. Denne test viser, at systemet fungerer som ønsket, selvom en anden konfiguration af vinkelberegningen havde været mere hensigtsmæssig for at holde systemets bruger inde for en knævinkel på $90-180^{\circ}$. Som systemet er nu, vil dets output kunne benyttes til et exoskelet inden for ovennævnte vinkelinterval over knæet. Dette kan gøres ved at anvende EMG-algoritmens digitale output til at få et exoskelet over knæleddet til at flekse, når output er +10, og ekstendere, når output er -10. En prototype i form af dette exoskelet er dog ikke udviklet. Det er derfor muligt at anvende systemet som et kontrolsystem til et exoskelet tiltænkt ALS-patienter under en squat-øvelse.

For at kunne styre samt anvende et exoskelet under en squat-øvelse benyttes derved signaler fra accelerometrene samt muskelsignaler fra rectus femoris. Ud fra signaler fra accelerometrene beregnes den samlede vinkel over knæet. Når denne vinkel befinder sig mellem  $90^{\circ}$ og $180^{\circ}$, vurderes det ud fra muskelaktiviteten om brugeren bevæger sig i en opadgående eller nedadgående retning. Herved vil kontrolsystemet kunne sende disse informationer trådløst til et exoskelet for således at kunne støtte ALS-patienter ved udførelse af en squat-øvelse.

På bagrund af ovenstående konkluderes det dermed, at problemformuleringen besvares. 