\section{Konklusion}

I dette projekt er der udviklet et digitalt system, der kan optage muskelsignaler fra rectus femoris samt signaler fra accelerometrene. Muskelsignalerne viser, hvorvidt muskelsignalerne er stigende eller faldende samtidig med, at signalerne fra accerelerometrene anvendes til beregning af vinklen over knæet under en squat-øvelse. Systemet er udviklet med henblik på at kunne hjælpe ALS-patienter til aflastning af muskulaturen omkring knæet. Systemets blokke, der består af opsamling af signaler, spændingsforsyning, ADC, digital filtrering, vinkelberegning, EMG-algoritme og trådløs kommunikation er designet, implementeret samt testet. Disse blokke evalueres ud fra testene for at vurdere, hvorvidt de opstillede kravspecifikationer for blokkene opfyldes. Ud fra dette ses det, at kravet for trådsløs kommunikation mellem mikrokontrolleren og LEGO mindstorm NXT ikke opfyldes, samt kravet om en maksimal forsinkelse på 100 ms ikke er opfyldt. Det kan ud fra evalueringen af testene konkluderes, at de resterende kravspecifikationer for de enkelte blokke er opfyldt. 

Det samlede system er herefter testet med et kendt inputsignal samt et inputsignal fra en bruger. Denne test viser, at systemet fungerer som ønsket. En prototype i form af et eksoskelet er ikke udviklet, hvorfor dette krav ikke er opfyldt. Det vil på baggrund af dette, på nuværende tidspunkt, ikke være muligt at anvende systemet som bodyaugmentation som hjælpemiddel til ALS-patienter. 

For at kunne styre samt anvende et eksoskelet under en squat-øvelse benyttes signalerne fra accelerometrene samt muskelsignalerne fra. Ud fra signalerne fra accelerometeret beregnes den samlede vinkel over knæet. Når denne vinkel befinder sig mellem  $90^{\circ}$ og $180^{\circ}$, vurderes det ud fra muskelaktiviteten om brugeren bevæger sig i en opadgående eller nedadgående retning. Herved vil disse informationer sendes trådløst til et eksoskelet for således at kunne anvende dette system som bodyaugmentation til ALS-patienter.