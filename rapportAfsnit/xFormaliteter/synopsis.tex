%Introduktion
The purpose of this project is to examine the possibility to control an exoskeleton during a squat exercise to support amyothrophic lateral sclerosis patients' muscles. This study consists of measurements from the muscle, rectus femoris, and two accelerometers. The muscle activity shows the patients movement during a squat. The inputs signals from the accelerometers are calculated to show the knee's angle. A digital system has been designed, implemented and tested to evaluate the system. The study is based on grey literature, articles, books and tests which has been analyzed and interpreted. It is not possible to use the system to help the patients currently since the prototype has not been developed. The test of the system shows that it possible to detect whether the user is moving upwards or downwards during a squat exercise, when the angle of the knee is between $90^{\circ}$ og $180^{\circ}$, by analysing the input signals from the rectus femoris and the accelerometres. 