%Introduktion
The purpose of this project is to examine the possibility to control an exoskeleton in order to support amyothrophic lateral sclerosis (ALS) patients during a squat exercise. 
In order to develop a control system, a system has been designed and implemented. A series of tests is conducted and documented to evaluate the performance of this system.   
The control system is based on measurements from the muscle rectus femoris and two accelerometers. These measurements allow detection of muscle activity and calculation of the knee's angle during a squat. 
Test of the system proves that it is possible to detect whether the user is moving upwards or downwards during a squat, when the knee is between $90^{\circ}$ and $180^{\circ}$.
Additional issues may be taken into account before ALS patients can benefit from the control system. The system must be combined with an exoskeleton to support ALS patients and reduce the strain on the muscles during a squat. 