\section{Forord}
Denne rapport og dette projekt er udarbejdet af gruppe 16gr4405, 4. semesters-studerende på Sundhedsteknologi, Aalborg Universitet. Projektet er udarbejdet i perioden mellem den 1. februar og den 27. maj 2016 og tager udgangspunkt i semestrets tema \textit{"Behandling af fysiologiske signaler"}. Yderligere er projektet udarbejdet på baggrund af projektforslaget \textit{"Udvikling af et EMG-baseret kontrolsystem til body-augmentation systemer"}. Ifølge studieordningen for Sundhedsteknologi på 4. semester har projektet følgende formål: \textit{"Med udgangspunkt i opnået viden, færdigheder og kompetencer på 3. semester arbejdes der med teori og metoder til opsamling og præsentation af signaler fra kroppen, men nu med fokus på digital signalbehandling og datakommunikation}. \citep{aalborguniversitet2014}

Af denne grund designes, implementeres og testes et system, hvor hovedfokus er signalbehandling samt datakommunikation. 
%Et EMG-baseret kontrolsystem er udviklet til aflastning at ALS-patienters lårmuskulatur. 

Der rettes tak til vejleder Steffen Frahm for god vejledning og et godt samarbejde. Herudover rettes der tak til John Hansen for hjælp til datakommunikation samt programmering. 