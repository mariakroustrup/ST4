\section*{Læsevejledning}
Projektrapporten er opdelt i otte kapitler samt tilhørende bilag. 
Det første kapitel indeholder en indledning til projektet samt den initierende problemstilling. 
Andet kapitel består af problemanalysen, der er udarbejdet på baggrund af den initierende problemstilling. 
Problemanalysen leder op til en projektafgrænsning samt problemformulering. 
Fra tredje til syvende kapitel beskrives problemløsningen, der består af systemudvikling, løsningsstrategi, teori og design, implementering og test af de enkelte blokke samt det samlede system. 
Det ottende og sidste kapitel består af syntese, der indeholder en diskussion, konklusion samt perspektivering af projektet. 
Dette efterfølges af litteraturliste og bilag. 

I dette projekt anvendes Vancouver-metoden til håndtering af kilder. 
Kilderne nummereres fortløbende i kantede parenteser. 
I litteraturlisten ses kilderne, der eksempelvis er angivet med forfatter, titel og årstal.  
Forkortelser i rapporten er første gang skrevet ud, efterfølgende er forkortelsen angivet i parentes. 
Herefter anvendes forkortelsen fremadrettet i rapporten.
 
Denne rapport er udarbejdet i \LaTeX. Herudover anvendes MATLAB til visualisering af grafer. Yderligere anvendes PSoC Creator til behandling af data samt programmering af systemet.  


\subsection*{Flowdiagram-håndtering} \label{sec:flowhaandtering}
For at kunne forstå og læse flowdiagrammerne i \autoref{sec:flow}, som anvendes i projektet, forklares betydningen af de forskellige figurer. Disse fremgår af \autoref{fig:flow}.

\begin{figure}[H]
\centering
\includegraphics[width=0.3\textwidth]{figures/flow}
\caption{Illustration af de anvendte figurer i flowdiagrammer. Herunder symboliserer cirklen en start- og stopfunktion, firkanten symoboliserer en beslutning og diamanten en midtvejsproces.}
\label{fig:flow}
\end{figure}

%\noindent
%Cirklen indikerer start og stop af funktion. Firkanten indikerer en beslutning. Diamant formen indikerer en midtvejsproces.