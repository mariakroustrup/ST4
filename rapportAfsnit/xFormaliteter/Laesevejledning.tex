\section*{Læsevejledning}
Projektrapporten er opdelt i 7 kapitler samt tilhørende bilag. Det første kapitel indeholder en indledning til projektet samt den initierende problemstilling. Andet kapitel består af problemanalysen, der er udarbejdet på baggrund af den initierende problemstilling. Problemanalysen leder op til en projektafgrænsning samt problemformulering. Fra tredje til sjette kapitel beskrives problemløsningen, der består af systemudvikling, løsningsstrategi, teori og design, implementering samt test af de enkelte blokke og det samlede system. Det syvende og sidste kapitel består af syntese, der indeholder en diskussion, konklusion samt perspektivering af projektet. Dette efterfølges af bilag samt litteraturlisten. 

I dette projekt anvendes Vancouver-metoden til refereringen af kilder. Kilderne referes som tal, der er omgivet af kantede parenteser. I litteraturlisten ses kilderne, der eksempelvis er angivet med forfatter, titel og årstal. Hvis kilden er angivet før et punktum, er der referet til den forrige sætning. Hvis kilden er angivet efter punktum, er der refereret til hele afsnittet. Forkortelser i rapporten er skrevet i en parentes første gang, hvorefter forkortelsen bliver anvendt i den resterende del af rapporten.
 
Denne projektrapport er udarbejdet i LaTex. Herudover er der anvendt MATLAB  til at visualisere diverse grafer. Yderligere er PSoC Creator anvendt til behandling af data samt programmering af systemet.  


\subsection*{Flowdiagram håndtering} \label{sec:flowhaandtering}
For at kunne forstå og læse flowdiagrammerne, som anvendes i projektet, forklares betydningen af de forskellige former. De forskellige former fremgår af \autoref{fig:flow}.

\begin{figure}[H]
\centering
\includegraphics[width=0.3\textwidth]{figures/flow}
\caption{Illustration af de anvendte former i flowcharts, der består af en cirkel, en firkant og en rombe.}
\label{fig:flow}
\end{figure}

\noindent
Cirklen indikerer start og stop af funktion. Firkanten indikerer en beslutning. Rude indikerer en midtvejs proces.