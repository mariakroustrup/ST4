\section{Læsevejledning}



\subsection{Kilde håndtering}
I dette projekt anvendes vancouver metoden, når der refereres til kilder. Kilderne referes som tal, der er omgivet af kantede parenteser. I litteraturlisten ses kilderne, der eksempelvis er angivet med forfatter, titel og årstal. Hvis kilden er angivet før et punktum, er der referet til sætningen. Hvis kilden er angivet efter punktum, er der refereret til afsnittet. 


\subsection{Flowdiagram håndtering} \label{sec:flowhaandtering}
For at kunne forstå og læse de flowdiagrammer som anvendes i projektet, fremgår korte forklaringer på hvad de forskellige former betyder. Formerne fremgår af \autoref{fig:flow}.

\begin{figure}[H]
\centering
\includegraphics[width=0.3\textwidth]{figures/flow}
\caption{Illustration af de anvendte former i flowcharts}
\label{fig:flow}
\end{figure}

\textbf{Figur 1:} Indikerer start og stop af funktion.
\textbf{Figur 2:} Indikerer en beslutning.
\textbf{Figur 3:} Indikerer en midtvejs proces.