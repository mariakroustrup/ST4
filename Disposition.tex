\subsection{Amytrofisk lateral sklerose}
Vi vil forklare generelt omkring ALS i forhold til det anatomiske og fysiologiske. I forhold til degenerering af nerver samt hvilken betydning dette får for patienten rent fysiologisk. Herudover vil vi forklare hvilke symptomer der opstår ved sygdom og beskrive hvordan progressionen er. Herudover vil vi beskrive at årsagen til ALS er ukendt, men eksperter mener at der ses et mønster rent genetisk. Vil vi prøve at finde studier som beskriver hvor mange der  mister evnen til at gå og se om der er en sammenhæng mellem hvornår de mister evnen til at gå. 

\subsection{Følger}


\subsection{Livskvalitet}
I dette afsnit vil vi komme ind på hvordan livskvaliteten for ALS patienter vurderes. Da der kun findes palliative behandlinger vurderes livskvaliteten ud fra overordnet livskvalitet og en sundhedsrelateret livskvalitet. Hvor der ses en problematik mellem disse to vurderinger, da den sociale støtte bl.a. bliver mere omfattende i takt med progressionen, og dermed kompenserer for de manglende fysiske egenskaber. På nuværende tidspunkt har vi ikke nogle view som beskriver livskvaliteten hos patienterne og hvilken betydning der har for patienterne at være 'bundet' til en kørestol eller respirator. 

\subsection{Teknologier og hjælpemidler}
I dette afsnit vil vi komme ind på de nuværende teknologier og hjælpemidler som patienter med ALS tilbyde og hvornår det er nødvendigt at anvende mere tilpassede teknologier. I forhold til de nuværende hjælpemidler vil vi komme ind på hvilke ulemper der i forhold til selvstændighed hos patienterne. Hvormed vi vil komme ind på hvordan denne selvstændighed, hvis det muliggøres at patienterne kan gå i længere tid i stedet for at være låst fast til en kørestol, dette kan muligvis gøre ved brug af body augmentation. En form for body augmentation er exoskeletet, som vi beskriver meget generelt og hvordan den anvendes, samt begrunder muliggørelsen af gang ved brug af studier som understøtter dette. 

\subsection{Afgrænsning}
I dette afsnit vil vi kort afgrænse os til knæet ved at opsumere nogle pointer fra de forrige afsnit.

\subsection{Knæleddet}
I dette afsnit vil vi forklare omkring hvor vigitgt knæleddet er for at mennesket er i stand til at gå. Dette vil blive beskrevet fysiologiske i forhold til muskler og led. Herudover vil det beskrives hvordan knæet muliggøre gang. Der vil derudover beskrives hvilken funktion knæet har ved udførelse af en sqaut øvelse.
